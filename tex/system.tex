\documentclass[12pt]{article}

\usepackage{comment}
 
\usepackage[noend]{algpseudocode}
\usepackage{algorithm}
\usepackage{float}
\usepackage{graphicx}
\usepackage[margin=.75in]{geometry} 
\usepackage{amsmath,amsthm,amssymb}
\usepackage{dsfont}
\usepackage{amsthm}
\usepackage{mathtools,amssymb}
\usepackage{wrapfig,caption,subcaption}
\allowdisplaybreaks

\newtheorem*{definition}{Definition}
\newtheorem*{question}{Question}
\newtheorem{theorem}{Theorem}
\newtheorem{proposition}[theorem]{Proposition}
\newtheorem{claim}[theorem]{Claim}
\newtheorem{lemma}[theorem]{Lemma}
\newtheorem{corollary}[theorem]{Corollary}
\newtheorem{conjecture}[theorem]{Conjecture}
 
\newcommand{\N}{\mathbb{N}}
\newcommand{\Z}{\mathbb{Z}}
\newcommand{\R}{\mathbb{R}}
\newcommand{\Rgz}{\mathbb{R}_{\ge 0}}

\newcommand{\ip}[2]{\left\langle{#1},{#2}\right\rangle}
\newcommand{\norm}[1]{\left\lVert{#1}\right\rVert}
\newcommand{\sizeof}[1]{\left\lvert{#1}\right\rvert}

\newcommand{\woloss}{without loss of generality }

\DeclareMathOperator*{\argmin}{arg\,min}
\DeclareMathOperator*{\argmax}{arg\,max}
\DeclareMathOperator*{\cone}{cone}
\DeclareMathOperator*{\hull}{hull}
\DeclareMathOperator*{\indif}{indif}

\newcommand{\1}[1]{\mathds{1}[{#1}]}
\renewcommand{\P}[1]{\mathds{P}\left[{#1}\right]}
\newcommand{\E}[1]{\mathds{E}\left[{#1}\right]}
\newcommand{\Var}[1]{\mathrm{Var}[{#1}]}

\newcommand{\unit}{\mathds{1}}
\newcommand{\lo}{\succ}

% \renewcommand{\thesubsection}{\thesection.\alph{subsection}}
% \renewcommand\thesubsection{\ \ (\alph{subsection})}


\begin{document}

% \renewcommand{\qedsymbol}{\filledbox}
 
\title{
  Dim Prefs
}
\author{
  Clay Thomas \\
  claytont@princeton.edu
}
\maketitle

\section{Lemmas}
  In this section, we set up the tools needed to reason about 
  $d$-dimensional preferences.

  \subsection{Lemmas based on the structure of $X$}
    These first few lemmas relate geometric properties of $X$ to
    limitations on the structure of $>_a$ for a specific, fixed $a$.

    The most simple way $X$ gives structure to preferences is if
    one outcome is better in all attributes.
    \begin{definition}
      Let $x,y\in X\subseteq \Rgz^d$.
      We say $x$ \emph{dominates} $y$, denoted $x\gg y$,
      if $x[k] > y[k]$ for each $k=1,\ldots,d$.
    \end{definition}
    \begin{proposition}
      If $x \gg y$, then $x >_a y$ for any nonzero $a\in \Rgz^d$.
    \end{proposition}

    When comparing different outcomes, a useful tool is
    the familiar geometric notion of a convex hull.
    Intuitively, if a preference weight does not like any of a set of options,
    it will not like any outcome in the hull of those options either.
    Thus, a point ``dominated by the hull'' of a set of options
    cannot be preferred to all those options.
    \begin{definition}
      For points $x_1,\ldots,x_n \in \R^d$, let $\hull(x_1,\ldots,x_n)
      = \{u_1x_1 + \ldots + u_nx_n | 0\le u_i\le 1, \sum_{i=1}^n u_i = 1\}$
      denote the convex hull of $x_1,\ldots,x_n$.
    \end{definition}
    \begin{lemma}\label{lem:agreementHull}
      Let $z,x_1,\ldots,x_k \in \Rgz^d$ and $a\in \Rgz^d \setminus \{0\}$.
      If $z >_a x_i$ for $i=1,\ldots,k$, then $z >_a w$
      for any $w\in \hull(x_1,\ldots,x_k)$.
    \end{lemma}
    \begin{proof}
      We have $\ip{a}{z} > \ip{a}{x_i}$ for each $i=1,\ldots,k$.
      If $w = u_1x_1+ \ldots + u_nx_n$ and $\sum_i u_i =1$, then
      $\ip{a}{w} = u_1\ip{a}{x_1}+\ldots+u_n\ip{a}{x_n}
      < u_1\ip{a}{z} + \ldots + u_n\ip{a}{z} = \ip{a}{z}$.
    \end{proof}

    \begin{proposition}\label{prop:domHull}
      Let $U, V \subseteq \Rgz^d$ be finite sets of points.
      Suppose that there exists $u\in \hull U$ and $v\in \hull V$
      such that $u \ll v$.
      Then no $a$ satisfies $u_i >_a v_i$ for each $i=1,\ldots, k$.
      % Suppose that there exists $w\in \hull(x_1,\ldots,x_k)$
      %     such that $z \gg w$.
      %     Then no $a$ satisfies $x_i >_a z$ for each $i=1,\ldots, k$.
    \end{proposition}
    \begin{proof}
      For contradiction, suppose such an $a$ exists.
      First, note $u_i >_a v \in \hull V$,
      then see that $\hull U \ni u >_a v$ as well.
      But $u\ll v$, so this is a contradiction.
    \end{proof}

    The converse of the last theorem also holds:
    \begin{proposition}
      Suppose that no $a\in \R_{>0}^d$ satisfies $u_i \ge_a v_j$ for each 
      $u_i \in U$ and $v_j\in V$.
      Then there exist $u\in\hull(U)$ and $v\in\hull(V)$ with $u\ll v$.
      NOTE: SHOULD USE THE ``SEMISTRICT`` DEFINITION OF << 
    \end{proposition}
    \begin{proof}
      If no $a$ has this property, then in particular the following linear
      program is infeasible:
      \begin{alignat*}{2}
        \text{min: } &
        \sum_{k=0}^d 0\cdot a_k \\
        \text{s.t. } &
        \begin{aligned}[t]
          \sum_{k=1}^d a_k (u^i_k - v^j_k)
            & \ge 0
            \quad \forall i\in U, j\in V
          \\ a_k & \ge 1 \quad \forall k
        \end{aligned}
      \end{alignat*}
      The dual of this linear program is
      \begin{alignat*}{2}
        \text{max: } &
          \sum_{i\in U, j\in V} 0\cdot b_{ij} + 1\cdot c_k \\
        \text{s.t. } &
        \begin{aligned}[t]
          \sum_{i\in U, j\in V} b_{ij}(u^i_k - v^j_k) + c_k
            & \le 0
            \quad \forall k
          \\ b_{ij} & \ge 0 \quad \forall i,j
          \\ c_k & \ge 0 \quad \forall i,j
        \end{aligned}
      \end{alignat*}
      This dual program is always feasible (with the all zeros solution)
      so by strong duality it must be unbounded.
      Take some solution and some $k$ with $c_k > 1$.
      At this solution, $b_{ij}$ are not all zero.
      Thus, take
      $\sum_{i,j} b_{ij}=B$, and consider:
      \[ \sum_{i\in U} \left(\sum_{j\in V}b_{ij}/B \right) u^i_k
        \le - \frac {c_k} B + \sum_{j\in V} \left(\sum_{i\in U}b_{ij}/B \right) v^j_k
        < \sum_{j\in V} \left(\sum_{i\in U}b_{ij} / B\right) v^j_k
      \]
      NOTE: THIS CANNOT BE CORRECT BECAUSE IT IMPLIES EG THAT ALL
      POINTS ARE MORE THAN 1 APART
    \end{proof}


    This motivates the following definition:
    \begin{definition}
      A set of points dominates, denoted $U \ll V$,
      when DEFINITION
    \end{definition}
\end{document}
