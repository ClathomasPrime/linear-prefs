\documentclass[12pt]{article}
 
\usepackage[noend]{algpseudocode}
\usepackage{algorithm}
\usepackage{algorithmicx}
\usepackage{float}
\usepackage{graphicx}
\usepackage[margin=1in]{geometry} 
\usepackage{amsmath,amsthm,amssymb}
\usepackage{dsfont}
\usepackage{subcaption}
\usepackage{amsthm}
\usepackage{mathtools,amssymb}
\allowdisplaybreaks

\newtheorem*{definition}{Definition}
\newtheorem*{question}{Question}
\newtheorem{theorem}{Theorem}
\newtheorem{proposition}[theorem]{Proposition}
\newtheorem{claim}[theorem]{Claim}
\newtheorem{lemma}[theorem]{Lemma}
\newtheorem{corollary}[theorem]{Corollary}
\newtheorem{conjecture}[theorem]{Conjecture}
 
\newcommand{\N}{\mathbb{N}}
\newcommand{\Z}{\mathbb{Z}}
\newcommand{\R}{\mathbb{R}}
\newcommand{\Rgz}{\mathbb{R}_{\ge 0}}

\newcommand{\ip}[2]{\left\langle{#1},{#2}\right\rangle}

\newcommand{\woloss}{without loss of generality }

\DeclareMathOperator*{\argmin}{arg\,min}
\DeclareMathOperator*{\argmax}{arg\,max}
\DeclareMathOperator*{\cone}{cone}
\DeclareMathOperator*{\hull}{hull}

\newcommand{\1}[1]{\mathds{1}[{#1}]}
\renewcommand{\P}[1]{\mathds{P}\left[{#1}\right]}
\newcommand{\E}[1]{\mathds{E}\left[{#1}\right]}
\newcommand{\Var}[1]{\mathrm{Var}[{#1}]}

\newcommand{\unit}{\mathds{1}}

% \renewcommand{\thesubsection}{\thesection.\alph{subsection}}
% \renewcommand\thesubsection{\ \ (\alph{subsection})}


\begin{document}

% \renewcommand{\qedsymbol}{\filledbox}
 
\title{Preferences Resulting From Weighted Sums}
\author{Clay Thomas\\
claytont@princeton.edu }

\maketitle

% \begin{align*}
%   a_1x_1 + a_2x_2 && > &&
%     a_1y_1  + a_2y_2 && > &&
%     a_1z_1  + a_2z_2 \\
%   b_1y_1  + b_2y_2 && > &&
%     b_1z_1  + b_2z_2  && > &&
%     b_1x_1  + b_2x_2 \\
%   c_1z_1  + c_2z_2  && > &&
%     c_1x_1  + c_2x_2 && > &&
%     c_1y_1  + c_2y_2 \\
% \end{align*}
% 
% % \begin{align*}
% %   -\frac{z_2 - x_2}{z_1 - x_1} > \frac{a_1}{a_2}
% %   >  -\frac{y_2 - z_2}{y_1 - z_1}
% %   \\
% %   \frac{b_1}{b_2} < -\frac{x_2 - y_2}{x_1 - y_1}
% %     && \frac{b_1}{b_2} > -\frac{y_2 - z_2}{y_1 - z_1}
% %     && \frac{b_1}{b_2} > -\frac{z_2 - x_2}{z_1 - x_1}
% %   \\
% %   \frac{c_1}{c_2} > -\frac{x_2 - y_2}{x_1 - y_1}
% %     && \frac{c_1}{c_2} < -\frac{y_2 - z_2}{y_1 - z_1}
% %     && \frac{c_1}{c_2} > -\frac{z_2 - x_2}{z_1 - x_1}
% %   \\
% % \end{align*}
% 
% \begin{align*}
%   x_1 - y_1 > -\frac{a_2}{a_1}(x_2 - y_2)
%     &&
%   y_1 - z_1 > -\frac{a_2}{a_1}(y_2 - z_2)
%     &&
%   z_1 - x_1 < -\frac{a_2}{a_1}(z_2 - x_2)
%   \\
%   x_1 - y_1 < -\frac{b_2}{b_1}(x_2 - y_2)
%     &&
%   y_1 - z_1 > -\frac{b_2}{b_1}(y_2 - z_2)
%     &&
%   z_1 - x_1 > -\frac{b_2}{b_1}(z_2 - x_2)
%   \\
%   x_1 - y_1 > -\frac{c_2}{c_1}(x_2 - y_2)
%     &&
%   y_1 - z_1  -\frac{c_2}{c_1}(y_2 - z_2)
%     &&
%   z_1 - x_1 > -\frac{c_2}{c_1}(z_2 - x_2)
%   \\
% \end{align*}
% 
% \begin{align*}
%   \frac{a_1}{a_2} (>)_{xy} -\frac{x_2 - y_2}{x_1 - y_1}
%     && \frac{a_1}{a_2} (>)_{yz} -\frac{y_2 - z_2}{y_1 - z_1}
%     && \frac{a_1}{a_2} (<)_{zx} -\frac{z_2 - x_2}{z_1 - x_1} \\
%   \frac{b_1}{b_2} (<)_{xy} -\frac{x_2 - y_2}{x_1 - y_1}
%     && \frac{b_1}{b_2} (>)_{yz} -\frac{y_2 - z_2}{y_1 - z_1}
%     && \frac{b_1}{b_2} (>)_{zx} -\frac{z_2 - x_2}{z_1 - x_1} \\
%   \frac{c_1}{c_2} (>)_{xy} -\frac{x_2 - y_2}{x_1 - y_1}
%     && \frac{c_1}{c_2} (<)_{yz} -\frac{y_2 - z_2}{y_1 - z_1}
%     && \frac{c_1}{c_2} (>)_{zx} -\frac{z_2 - x_2}{z_1 - x_1} \\
% \end{align*}

\section{Motivation}
  Suppose students are told to rank different schools they would like to
  get into. The preferences of the students are likely correlated in some
  way due to an inherent notion of the quality of different schools.
  One way to define such a correlation is to assume there is some underlying
  space of quality along different attributes (e.g. STEM education vs
  liberal arts education) and the students' preference
  are determined by these attributes.
  The simplest instance of this is for each student to rank schools
  according to a weighted sum of the different attributes of the school.

  We want to study the inherent complexity of the collection of preferences
  that result from these procedure, as a function of the number of attributes
  the schools have. In other words, what sort of correlation arises in the
  preferences of students in this model?

  % as a weight vector $a = (a[1], \ldots, a[d])$, with each $a[k]\ge 0$.
  % Then every school $x$ is a vector $x = (x[1], \ldots, x[d])$ (with $x[k]\ge 0$)
  % of quality along the $d$ different attributes.
  % Student $a$ prefers

\section{Definitions}

  Let $\{x_1,\ldots,x_n\} = X\subseteq \Rgz^d$ be any set of points
  with nonnegative coordinates.
  Given any nonzero $a\in \Rgz^d$, define a total order $>_a$ on $X$ as follows:
  $x_i >_a x_j$ if and only if $\ip{a}{x_i} > \ip{a}{x_j}$, or
  $\ip{a}{x} = \ip{a}{y}$ and $i > j$ (note\footnote{
    It's subtly important that ties are broken in the same way by every $a$.
    Otherwise, if $X$ was $n$ copies of the same point, any preference
    would be possible.
  }).
  % Let $R(X)$ denote the set of linear orders on $X$.
  Let $P(X) = \{ >_a | a\in\Rgz^d \setminus \{0\} \}$.
  Thus, $X$ represents the set of schools,
  the vectors $a$ represent the preference weights of students,
  and $P(X)$ denotes the set of all possible preferences of schools.

  \begin{question}
    As a function of $d$ and $n$, how ``rich'' can $P(X)$ be
    (and what is the right notion of ``richness'')?
  \end{question}

  Some observations:
  \begin{itemize}
    \item If $d=1$, then $|P(X)| = 1$, i.e. students preferences are completely
      determined by the underlying set $X$.
    \item If $d=n$, then every linear preference on $X$ can occur in $P(X)$.
      \begin{proof}
        Let $X = \{e_i\}$ simply be the standard basis vectors.
        To induce an ordering $i_1, i_2, \ldots, i_n$, just create a preference
        vector $a$ which gives weight $1/k$ to coordinate $i_k$.
      \end{proof}
  \end{itemize}
  The above hints that there should be some sort of continuum between $d=1$
  and $d=n$ of how complex the set $P(X)$ can be.

\section{Lemmas}

  Here are some simple ways that the structure of $X$ induces structure on $P(X)$:

  \begin{definition}
    Let $x,y\in X\subseteq \Rgz^d$.
    We say $x$ \emph{dominates} $y$, denoted $x\gg y$,
    if $x[k] > y[k]$ for each $k=1,\ldots,d$.
  \end{definition}
  \begin{proposition}
    If $x \gg y$, then $x >_a y$ for any nonzero $a\in \Rgz^d$.
  \end{proposition}
  \begin{proof}
    Simply observe $a_i x_i \ge a_i y_i$ for each $i=1,\ldots, d$.
    Because $a\ne 0$, there is also some $i$ where $a_i x_i > a_i y_i$.
  \end{proof}


  \begin{definition}
    For points $x_1,\ldots,x_n \in \R^d$, let $\hull(x_1,\ldots,x_n)
    = \{u_1x_1 + \ldots + u_nx_n | 0\le u_i\le 1, \sum_{i=1}^n u_i = 1\}$
    denote the convex hull of $x_1,\ldots,x_n$.
  \end{definition}
  \begin{lemma}
    Let $z,x_1,\ldots,x_k \in \Rgz^d$ and $a\in \Rgz^d \setminus \{0\}$.
    If $z >_a x_i$ for $i=1,\ldots,k$, then $z >_a w$
    for any $w\in \hull(x_1,\ldots,x_k)$.
  \end{lemma}
  \begin{proof}
    We have $\ip{a}{z} > \ip{a}{x_i}$ for each $i=1,\ldots,k$.
    If $w = u_1x_1+ \ldots + u_nx_n$ and $\sum_i u_i =1$, then
    $\ip{a}{w} = u_1\ip{a}{x_1}+\ldots+u_n\ip{a}{x_n}
    < u_1\ip{a}{z} + \ldots + u_n\ip{a}{z} = \ip{a}{z}$
  \end{proof}
  \begin{proposition}
    Let $z,x_1,\ldots,x_k \in \Rgz^d$.
    Suppose that there exists $w\in \hull(x_1,\ldots,x_k)$
    such that $w \gg z$.
    Then no $a$ satisfies $z >_a x_i$ for each $i=1,\ldots, k$.
  \end{proposition}
  \begin{proof}
    For contradiction, suppose such an $a$ exists.
    Then $z >_a w$ as well. However, because $w \gg z$,
    this is a contradiction.
  \end{proof}

  Let's look at how preferences are related when different weight vectors $a$
  are related.
  \begin{definition}
    For any vectors $a_1,\ldots, a_k \in \Rgz^d$, define
    $\cone(a_1,\ldots,a_k) = \{ u_1a_1 + \ldots + u_ka_k | u_i\ge 0 \forall j\}$.
    That is, the convex cone of vectors defined by $a_1,\ldots,a_k$.
  \end{definition}
  \begin{proposition}
    Suppose that for preference weights $a_1,\ldots, a_k$,
    we have $x >_{a_1} y, \ldots, x >_{a_k} y$.
    Then for any nonzero $b\in \cone(a_1,\ldots, a_k)$,
    $x >_b y$ as well.
  \end{proposition}
  \begin{proof}
    Let $b = u_1a_1+ \ldots + u_ka_k$.
    We get $\ip{b}{x} = u_1\ip{a_1}{x} + \ldots + u_k\ip{a_k}{x}
    > u_1\ip{a_1}{y} + \ldots + u_k\ip{a_k}{y} = \ip{b}{y}$,
    as desired.
  \end{proof}

\section{Impossibility Results for $d=2$}

  \begin{proposition}
    If $d=2$, then the preferences
    \begin{align*}
      x >_a y >_a z \\
      y >_b z >_b x\\
      z >_c x >_c y\\
    \end{align*}
    cannot occur for $x,y,z\in X\subseteq \Rgz^2$.
  \end{proposition}
  \begin{proof}
    Suppose the above preferences exist.
    Because $d=2$, one of $a,b,c$ must be contained in the
    convex cone of the other two, i.e. $d \in \{a,b,c\}$
    with $d \in \cone(e,f)$ where $\{e,f\} = \{a,b,c\} \setminus \{d\}$.
    (This is visually very obvious: given three different
    vectors in two dimensions, one must lie between the other two.
    You can prove it by considering the angle of the vectors from the $x$ axis).
    Any pair of the above preferences has exactly one pair
    among $x,y,z$ which they rank the same, so
    $u >_e v$ and $u >_f v$ for some $u,v$.
    However, this pair is reversed in the remaining preference, i.e. $u <_d v$.
  \end{proof}

  More generally, we have the following:
  \begin{proposition}
    Let $d=2$ and $|X| = 3$. Then it is not possible
    for every point to be ranked highest by some preference,
    and ranked lowest by some other preference.
  \end{proposition}
  \begin{proof}
    Let $\{x,y,z\}$ be any set of distinct points in $\Rgz^2$.
    Because $d=2$, there must exist some $p\in \{x,y,z\}$
    such that $p \gg w$ or $p \ll w$ for some
    $w \in \hull(r,s)$, where $\{r,s\} = \{x,y,z\} \setminus \{p\}$
    (again, a proof by picture is best here, but it can be formally proven).
    But then, no $e$ satisfies $p >_e r$ and $p >_e s$,
    OR no $e$ satisfies $p <_e r$ and $p <_e s$.
  \end{proof}
  \begin{corollary}
    If $d=2$, then the preferences
    \begin{align*}
      x >_a y >_a z \\
      x >_b z >_b y \\
      y >_c z >_c x \\
      z >_d y >_d x \\
    \end{align*}
    cannot occur for $x,y,z\in X\subseteq \Rgz^2$.
  \end{corollary}

  It turns out that the above two patterns are,
  up to relabeling, the \emph{only} impossible subsets
  of preferences with $|X|=3$.
  We believe these patterns extend to higher dimensions,
  but the proofs above do not generalize because of the geometry involved.

  %\section{Cycles}

  % The above set of preferences intuitively forms a ``cycle'',
  % where each preference is ``shifted'' over by one. Formally,
  % \begin{definition}
  %   A $k$-cycle is a set of points $x_1,\ldots, x_k\in \Rgz^d$ and weights 
  %   $a_1,\ldots, a_k \in \Rgz^d$ such that for each $i=1,\ldots,k$,
  %   \begin{align*}
  %     x_i >_{a_i} x_{i+1} >_{a_i} x_{i+2} >_{a_i} \ldots >_{a_i} x_{i-1}
  %   \end{align*}
  %   (where indices wrap around after $k$).
  % \end{definition}

  % \begin{conjecture}
  %   If $X \subseteq \Rgz^d$, then a $(d+1)$-cycle can never appear in $P(X)$.
  % \end{conjecture}

  % The second part of the proof from $d=2$ generalizes:
  % in a $d+1$ cycle, any subset of $d$ preferences has a consistent preference
  % on some pair, and this preference is broken in the remaining
  % element of the cycle.
  % However, the first part is not true: it is easy to construct $4$ vectors in
  % $3$ dimensions such that none lies in the convex cone of the others.

  % This conjecture has been checked by randomly placing $d+1$ points 
  % in $d$ dimensional space many times, for $d=3,4$.

  % It may be that in some sense, $(d+1)$-cycles are the \emph{only}
  % set of preferences that are impossible with $d$ dimensions.
  % For example, there are two different (up to relabeling) ways for
  % a collection of preferences on three items to not contain a cycle.
  % Writing a linear preference as a simple ordered (most preferable
  % to least), the preference sets are $\{ 123, 132, 312, 321\}$ and
  % $\{ 123, 231, 213, 321 \}$. Both are realizable with $d=2$
  % (the first via points $\{(1,0), (0.4,0.4), (0,1) \}$
  % and the second with $\{(1,0), (0.6,0.6), (0,1)\}$).

  % When $d=3$, randomly searching over $X$ for the largest $|P(X)|$
  % yield a maximum size of $18$, and those sets exactly correspond
  % to removing one element from each of the $6$ cycles on $4$ points.
  % A variety of different ways of doing this are possible,  but I'm
  % not yet sure if all of them result.

\section{Experimental Counts}
  The following gives upper bounds on the maximal number
  of preferences $|P(X)|$ as a function of $n$ and $d$.
  I've only included entries if those points seem relatively
  close to the true maximum (i.e. when things seem pretty stable
  to checking more preference vectors or doing more trials).

  \begin{tabular}{c | c c c c c c c}
    n & d=2 & d=3 & d=4 \\
    \hline
    3  &  4   & 6  & 6  \\
    4  &  7   & 18 & 24 \\
    5  &  10  & 41 & 85 \\
    6  &  15  & 87  \\
    7  &  20  & 121 \\
    8  &  24  \\
    9  &  28  \\
    10 &  39  \\
    11 &  43  \\
    12 &  48  \\
  \end{tabular}

% \section{An (Almost) Semidefinite Program Formulation}
%   In this section, we formalize a decision problem for whether a set
%   of preferences can arise from a set of points in $\Rgz^d$,
%   then show how this can be reduced to checking the feasibility
%   of a semidefinite program.
% 
%   Consider the decision problem given by: \\
%   \textbf{Input:} $d, n$, and a collection $P = \{ >_1, >_2, \ldots, >_k\}$
%   of linear orders on $[n]$ \\
%   \textbf{Output:} YES if there exists a set of $n$ points $X\subseteq \Rgz^d$
%   in $d$ dimensions such that $P \subseteq P(X)$
%   (according to some labeling of points in $X$ with $[n]$),
%   and NO otherwise.
% 
%   This is equivalent to asking whether there exist
%   vectors 

\end{document}


% \clearpage
% \appendix
% \section{Figures}
%   \begin{figure}[ht]
% 
%     We can prove the following structural impossibility results
%     on the set of preferences we consider.
% 
%     \begin{minipage}[b]{0.25\linewidth}
%       \centering
%       \begin{align*}
%         x >_a y >_a z \\
%         y >_b z >_b x \\
%         z >_c x >_c y \\
%       \end{align*}
%       \caption{This preference set is impossible if $d=2$.}
%       \label{fig:figure1}
%     \end{minipage}
%     \hspace{0.5cm}
%     \begin{minipage}[b]{0.25\linewidth}
%       \centering
%       \begin{align*}
%         x >_a y >_a z \\
%         x >_b z >_b y \\
%         y >_c z >_c x \\
%         z >_d y >_d x \\
%       \end{align*}
%       \caption{This preference set is impossible if $d=2$.}
%       \label{fig:figure2}
%     \end{minipage}
%     \hspace{0.5cm}
%     \begin{minipage}[b]{0.25\linewidth}
%       \centering
%       \begin{align*}
%         w >_a x >_a y >_a z \\
%         x >_b y >_b z >_b w \\
%         y >_c z >_c w >_c x \\
%         z >_d w >_d x >_d y \\
%       \end{align*}
%       \caption{This preference set is impossible if $d=3$.}
%       \label{fig:cycle4}
%     \end{minipage}
%   \end{figure}
% 
%   The following table gives approximate upper bounds for
%   the number of preferences possible on $n$ options in $d$
%   dimensions. The pattern seems like it might be something
%   kinda like $O(n^{d-1})$, which is much less than the full
%   set of preferences, which has size $n!$.
%   \begin{tabular}{c | c c }
%     n & d=2 & d=3 \\
%     \hline
%     3  &  4   & 6   \\
%     4  &  7   & 18  \\
%     5  &  10  & 41  \\
%     6  &  15  & 87  \\
%     7  &  20  & 121 \\
%     8  &  24  \\
%     9  &  28  \\
%   \end{tabular}
% 
%       % \begin{align*}
%       %   x >_1 y >_1 z \\
%       %   y >_2 z >_2 x \\
%       %   z >_3 x >_3 y \\
%       % \end{align*}
%       % \caption{\textsc{Cycle}}
%       % \begin{align*}
%       %   x >_1 y >_1 z \\
%       %   x >_2 z >_2 y \\
%       %   y >_3 z >_3 x \\
%       %   z >_4 y >_4 x \\
%       % \end{align*}
% 
