\documentclass[12pt]{article}
 
\usepackage[noend]{algpseudocode}
\usepackage{algorithm}
\usepackage{algorithmicx}
\usepackage{float}
\usepackage{graphicx}
\usepackage[margin=1in]{geometry} 
\usepackage{amsmath,amsthm,amssymb}
\usepackage{dsfont}
\usepackage{subcaption}
\usepackage{amsthm}
\usepackage{mathtools,amssymb}
\usepackage{wrapfig}
\allowdisplaybreaks

\newtheorem*{definition}{Definition}
\newtheorem*{question}{Question}
\newtheorem{theorem}{Theorem}
\newtheorem{proposition}[theorem]{Proposition}
\newtheorem{claim}[theorem]{Claim}
\newtheorem{lemma}[theorem]{Lemma}
\newtheorem{corollary}[theorem]{Corollary}
\newtheorem{conjecture}[theorem]{Conjecture}
 
\newcommand{\N}{\mathbb{N}}
\newcommand{\Z}{\mathbb{Z}}
\newcommand{\R}{\mathbb{R}}
\newcommand{\Rgz}{\mathbb{R}_{\ge 0}}

\newcommand{\ip}[2]{\left\langle{#1},{#2}\right\rangle}

\newcommand{\woloss}{without loss of generality }

\DeclareMathOperator*{\argmin}{arg\,min}
\DeclareMathOperator*{\argmax}{arg\,max}
\DeclareMathOperator*{\cone}{cone}
\DeclareMathOperator*{\hull}{hull}

\newcommand{\1}[1]{\mathds{1}[{#1}]}
\renewcommand{\P}[1]{\mathds{P}\left[{#1}\right]}
\newcommand{\E}[1]{\mathds{E}\left[{#1}\right]}
\newcommand{\Var}[1]{\mathrm{Var}[{#1}]}

\newcommand{\unit}{\mathds{1}}

% \renewcommand{\thesubsection}{\thesection.\alph{subsection}}
% \renewcommand\thesubsection{\ \ (\alph{subsection})}


\begin{document}

% \renewcommand{\qedsymbol}{\filledbox}

\title{Preferences Resulting From Weighted Sums}
\author{
Clay Thomas\\
claytont@princeton.edu 
\and
Yufei Zheng\\
yufei@cs.princeton.edu
}

\maketitle

\section{Motivation}
  Suppose students are told to rank different schools they would like to
  get into. The preferences of the students are likely correlated in some
  way due to an inherent notion of the quality of different schools.
  One way to define such a correlation is to assume there is some underlying
  space of quality along different attributes (e.g. STEM education vs
  liberal arts education) and the students' preference
  are determined by these attributes.
  The simplest instance of this is for each student to rank schools
  according to a weighted sum of the different attributes of the school.

  We want to study the inherent complexity of the collection of preferences
  that result from these procedure, as a function of the number of attributes
  the schools have. In other words, what sort of correlation arises in the
  preferences of students in this model?


% \section{An (Almost) Semidefinite Program Formulation}
%   In this section, we formalize a decision problem for whether a set
%   of preferences can arise from a set of points in $\Rgz^d$,
%   then show how this can be reduced to checking the feasibility
%   of a semidefinite program.
% 
%   Consider the decision problem given by: \\
%   \textbf{Input:} $d, n$, and a collection $P = \{ >_1, >_2, \ldots, >_k\}$
%   of linear orders on $[n]$ \\
%   \textbf{Output:} YES if there exists a set of $n$ points $X\subseteq \Rgz^d$
%   in $d$ dimensions such that $P \subseteq P(X)$
%   (according to some labeling of points in $X$ with $[n]$),
%   and NO otherwise.
% 
%   This is equivalent to asking whether there exist
%   vectors 

\clearpage
\appendix
\section{Figures}
  \begin{figure}[ht]

    We can prove the following structural impossibility results
    on the set of preferences we consider.

    \begin{minipage}[b]{0.25\linewidth}
      \centering
      \begin{align*}
        x >_a y >_a z \\
        y >_b z >_b x \\
        z >_c x >_c y \\
      \end{align*}
      \caption{This preference set is impossible if $d=2$.}
      \label{fig:figure1}
    \end{minipage}
    \hspace{0.5cm}
    \begin{minipage}[b]{0.25\linewidth}
      \centering
      \begin{align*}
        x >_a y >_a z \\
        x >_b z >_b y \\
        y >_c z >_c x \\
        z >_d y >_d x \\
      \end{align*}
      \caption{This preference set is impossible if $d=2$.}
      \label{fig:figure2}
    \end{minipage}
    \hspace{0.5cm}
    \begin{minipage}[b]{0.25\linewidth}
      \centering
      \begin{align*}
        w >_a x >_a y >_a z \\
        x >_b y >_b z >_b w \\
        y >_c z >_c w >_c x \\
        z >_d w >_d x >_d y \\
      \end{align*}
      \caption{This preference set is impossible if $d=3$.}
      \label{fig:cycle4}
    \end{minipage}
  \end{figure}

  The following table gives approximate upper bounds for
  the number of preferences possible on $n$ options in $d$
  dimensions. The pattern seems like it might be something
  kinda like $O(n^{d-1})$, which is much less than the full
  set of preferences, which has size $n!$.
  \begin{tabular}{c | c c }
    n & d=2 & d=3 \\
    \hline
    3  &  4   & 6   \\
    4  &  7   & 18  \\
    5  &  10  & 41  \\
    6  &  15  & 87  \\
    7  &  20  & 121 \\
    8  &  24  \\
    9  &  28  \\
  \end{tabular}

\end{document}
