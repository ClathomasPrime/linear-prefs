\documentclass[12pt]{article}
 
\usepackage[noend]{algpseudocode}
\usepackage{algorithm}
\usepackage{algorithmicx}
\usepackage{float}
\usepackage{graphicx}
\usepackage[margin=0.5in]{geometry} 
\usepackage{amsmath,amsthm,amssymb}
\usepackage{dsfont}
\usepackage{subcaption}
\usepackage{amsthm}
\usepackage{mathtools,amssymb}
\usepackage{wrapfig}
\usepackage[font=small,skip=-20pt]{caption}
\allowdisplaybreaks

\newtheorem*{definition}{Definition}
\newtheorem*{question}{Question}
\newtheorem{theorem}{Theorem}
\newtheorem{proposition}[theorem]{Proposition}
\newtheorem{claim}[theorem]{Claim}
\newtheorem{lemma}[theorem]{Lemma}
\newtheorem{corollary}[theorem]{Corollary}
\newtheorem{conjecture}[theorem]{Conjecture}
 
\newcommand{\N}{\mathbb{N}}
\newcommand{\Z}{\mathbb{Z}}
\newcommand{\R}{\mathbb{R}}
\newcommand{\Rgz}{\mathbb{R}_{\ge 0}}

\newcommand{\ip}[2]{\left\langle{#1},{#2}\right\rangle}

\newcommand{\woloss}{without loss of generality }

\DeclareMathOperator*{\argmin}{arg\,min}
\DeclareMathOperator*{\argmax}{arg\,max}
\DeclareMathOperator*{\cone}{cone}
\DeclareMathOperator*{\hull}{hull}

\newcommand{\1}[1]{\mathds{1}[{#1}]}
\renewcommand{\P}[1]{\mathds{P}\left[{#1}\right]}
\newcommand{\E}[1]{\mathds{E}\left[{#1}\right]}
\newcommand{\Var}[1]{\mathrm{Var}[{#1}]}

\newcommand{\unit}{\mathds{1}}

% \renewcommand{\thesubsection}{\thesection.\alph{subsection}}
% \renewcommand\thesubsection{\ \ (\alph{subsection})}


\begin{document}

% \renewcommand{\qedsymbol}{\filledbox}

\title{
  \vspace{-0.5in}
  Voting With $d$-Dimensional Preferences
}
\date{}
\author{
  Clay Thomas\\
  claytont@princeton.edu 
  \and
  Yufei Zheng\\
  yufei@cs.princeton.edu
}


\maketitle


  \begin{wrapfigure}{r}{0.2\textwidth}
    \begin{center}
      \vspace{-0.5in}
      \begin{align*}
        1 > 2 > 3 \\
        2 > 3 > 1 \\
        3 > 1 > 2 \\
      \end{align*}
      \caption{\textsc{Cycle}}
      \vspace{-0.25in}
      \begin{align*}
        1 > 2 > 3 \\
        1 > 3 > 2 \\
        3 > 2 > 1 \\
        2 > 3 > 1 \\
      \end{align*}
      \caption{\textsc{Sandwich}}
      \vspace{-0.25in}
      \begin{align*}
        1 > 2 > 3 \\
        1 > 3 > 2 \\
        3 > 2 > 1 \\
      \end{align*}
      \caption{\\ \textsc{LeastFavorite}}
      \vspace{-0.25in}
      \begin{align*}
        1 > 2 > 3 > 4 \\
        1 > 2 > 4 > 3 \\
        2 > 1 > 3 > 4 \\
        2 > 1 > 4 > 3 \\
      \end{align*}
      \caption{\textsc{FlipFlop}}
      \vspace{-0.25in}
      \begin{align*}
        1 > 2 > 3 > 4 \\
        4 > 3 > 2 > 1 \\
      \end{align*}
      \caption{\textsc{Reverse}}
    \end{center}
  \end{wrapfigure}

  \emph{Social choice theory} is the field of study
  concerned with making collective decisions. A typical formulation
  is to study ``voting schemes'', i.e.
  functions $f$ which take as input a list of \emph{linear
  preferences} on $[n]$ (that is, total orders on $[n]$) and outputs a
  ``group decision'' $f(>_1,\ldots,>_m) \in [n]$.
  When $n = 2$, majority rule constitutes an excellent choice of $f$.
  Unfortunately when $n>2$, social choice functions are
  plagued by impossibility results,
  such as Arrow's Impossibility Theorem (every ``unanimous'' voting scheme
  satisfying ``independence of irrelevant alternatives'' is a ``dictatorship'')
  and the Gibbard-Satterthwaite Theorem (every surjective 
  ``strategy-proof'' voting scheme is a dictatorship).

  To get around the issues with general voting schemes, theorists
  consider \emph{restricted classes of preferences},
  for example \emph{single peaked preferences}.
  Consider a set $P = \{>_i\}_{i}$ of linear preferences on $[n]$.
  We say $P$ is single peaked when there exists a set
  $Y = \{y_1,\ldots,y_n\} \subseteq [0,1]$ of distinct numbers such that,
  for each $>_i \in P$, there exists a ``peak'' $p_i \in [0,1]$
  where: if $p_i < y_k < y_\ell$, then $k >_i \ell$,
  and if $y_k < y_\ell < p_i$, then $k <_i \ell$.
  This corresponds to voters having a ``favorite point'' (i.e.
  their peak), and their preference for an alternative decreases
  as you move away from the peak.
  Single peaked preferences are dramatically better behaved
  than general preferences.
  If $P$ is single peaked, then there exist voting schemes
  $f : P^m \to [n]$ which are ``strategy-proof'',
  ``Pareto-optimal'' (a stronger
  version of unanimity), and ''anonymous'' (a stronger version
  of not being a dictatorship).
  This voting scheme essentially takes the medians of the ``peaks''.

  We say $P$ is \emph{$d$-dimensional} if there exists a set
  $X = \{x_1,\ldots,x_n\} \subseteq \R_{\ge 0}^d$ such that, for each $>_i \in P$,
  there exists a ``preference weight'' $a_i$ such that
  $k >_i \ell$ if and only if $\ip{a_i}{x_k} > \ip{a_i}{x_\ell}$.
  This corresponds to voters choosing options based on a simple weighted
  sum of the attributes of these options.
  \textbf{The goal of our project is to study voting schemes
  for $d$-dimensional preferences}.

  When $d=1$, the $1$-dimensional preference sets are trivial,
  i.e. they contain only one preference.
  The class of single peaked preferences and $2$-dimensional preferences
  seem vaguely related.
  For example, both preference sets \textsc{Cycle} (a common counterexample
  for certain voting schemes having nice properties) and \textsc{Sandwich},
  shown to the right, are neither single peaked nor $2$-dimensional.
  However, we note that neither class is a subset of the other:
  for example, \textsc{LeastFavorite} is $2$-dimensional but not single peaked,
  and \textsc{FlipFlop} is single peaked but not two dimensional.
  Finally, \textsc{Reverse} gives a nontrivial preference set which
  is both single peaked and $2$-dimensional.
  The proofs that certain preference sets fail to be $d$-dimensional
  are interesting and novel, and we regard the combinatorial classification
  of $d$-dimensional preference sets to be of independent interest.

  We conjecture good voting schemes are possible $2$-dimensional preferences.
  For $d\ge 3$, the situation appears less nice.
  Any set of preferences on $3$ options is $3$-dimensional,
  and $3$ options is already sufficient to prove negative
  results such as Arrow's Theorem.
  However, there still may be interesting things to say in this case.
  For all values of $d$, the rich space of questions relating to
  voting schemes will almost certainly lead to good avenues of investigation,
  if not especially impressive results \cite{MattConversation}.

  \clearpage
  \bibliography{weightedPref}{}
  \bibliographystyle{plain}

\end{document}
