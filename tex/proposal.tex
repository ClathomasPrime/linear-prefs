\documentclass[12pt]{article}
 
\usepackage[noend]{algpseudocode}
\usepackage{algorithm}
\usepackage{algorithmicx}
\usepackage{float}
\usepackage{graphicx}
\usepackage[margin=0.75in]{geometry} 
\usepackage{amsmath,amsthm,amssymb}
\usepackage{dsfont}
\usepackage{subcaption}
\usepackage{amsthm}
\usepackage{mathtools,amssymb}
\usepackage{wrapfig}
\usepackage[font=small,skip=-20pt]{caption}
\allowdisplaybreaks

\newtheorem*{definition}{Definition}
\newtheorem*{question}{Question}
\newtheorem{theorem}{Theorem}
\newtheorem{proposition}[theorem]{Proposition}
\newtheorem{claim}[theorem]{Claim}
\newtheorem{lemma}[theorem]{Lemma}
\newtheorem{corollary}[theorem]{Corollary}
\newtheorem{conjecture}[theorem]{Conjecture}
 
\newcommand{\N}{\mathbb{N}}
\newcommand{\Z}{\mathbb{Z}}
\newcommand{\R}{\mathbb{R}}
\newcommand{\Rgz}{\mathbb{R}_{\ge 0}}

\newcommand{\ip}[2]{\left\langle{#1},{#2}\right\rangle}

\newcommand{\woloss}{without loss of generality }

\DeclareMathOperator*{\argmin}{arg\,min}
\DeclareMathOperator*{\argmax}{arg\,max}
\DeclareMathOperator*{\cone}{cone}
\DeclareMathOperator*{\hull}{hull}

\newcommand{\1}[1]{\mathds{1}[{#1}]}
\renewcommand{\P}[1]{\mathds{P}\left[{#1}\right]}
\newcommand{\E}[1]{\mathds{E}\left[{#1}\right]}
\newcommand{\Var}[1]{\mathrm{Var}[{#1}]}

\newcommand{\unit}{\mathds{1}}

% \renewcommand{\thesubsection}{\thesection.\alph{subsection}}
% \renewcommand\thesubsection{\ \ (\alph{subsection})}


\begin{document}

% \renewcommand{\qedsymbol}{\filledbox}

\title{
  \vspace{-0.5in}
  Voting With $d$-Dimensional Preferences
}
\date{}
\author{
  Clay Thomas\\
  claytont@princeton.edu 
  \and
  Yufei Zheng\\
  yufei@cs.princeton.edu
}


\maketitle


  \begin{wrapfigure}{r}{0.2\textwidth}
    \begin{center}
      \vspace{-0.5in}
      \begin{align*}
        1 > 2 > 3 \\
        2 > 3 > 1 \\
        3 > 1 > 2 \\
      \end{align*}
      \caption{\textsc{Cycle}}
      \vspace{-0.25in}
      \begin{align*}
        1 > 2 > 3 \\
        1 > 3 > 2 \\
        3 > 2 > 1 \\
        2 > 3 > 1 \\
      \end{align*}
      \caption{\textsc{Sandwich}}
      \vspace{-0.25in}
      \begin{align*}
        1 > 2 > 3 \\
        1 > 3 > 2 \\
        3 > 2 > 1 \\
      \end{align*}
      \caption{\textsc{LeastFav}}
      \vspace{-0.25in}
      \begin{align*}
        1 > 2 > 3 > 4 \\
        1 > 2 > 4 > 3 \\
        2 > 1 > 3 > 4 \\
        2 > 1 > 4 > 3 \\
      \end{align*}
      \caption{\textsc{FlipFlop}}
      \vspace{-0.25in}
      \begin{align*}
        1 > 2 > 3 > 4 \\
        4 > 3 > 2 > 1 \\
      \end{align*}
      \caption{\textsc{Reverse}}
    \end{center}
  \end{wrapfigure}

  \emph{Social choice theory} is the field of study
  concerned with making collective decisions. A typical formulation
  is to study ``social choice functions'', i.e.
  functions $f$ which take as input a list of \emph{linear
  preferences} on $[n]$ (that is, total orders on $[n]$) and outputs a
  ``group decision'' $f(>_1,\ldots,>_m) \in [n]$.
  When $n = 2$, majority rule constitutes an excellent choice of $f$.
  Unfortunately when $n>2$, social choice functions are
  plagued by impossibility results,
  such as Arrow's Impossibility Theorem (every ``unanimous'' voting scheme
  satisfying ``independence of irrelevant alternatives'' is a ``dictatorship'')
  and the Gibbard-Satterthwaite Theorem (every surjective 
  ``strategy-proof'' voting scheme is a dictatorship) 
  \cite{AgtBookMechDesignInto}.

  To get around the issues with general voting schemes, theorists
  consider \emph{restricted classes of preferences},
  for example \emph{single peaked preferences}.
  Consider a set $P = \{>_i\}_{i=1}^K$ of linear preferences on $[n]$.
  We say $P$ is single peaked when there exists a set
  $Y = \{y_1,\ldots,y_n\} \subseteq [0,1]$ of distinct numbers such that,
  for each $>_i \in P$, there exists a ``peak'' $p_i \in [0,1]$
  where: if $p_i < y_k < y_\ell$, then $k >_i \ell$,
  and if $y_k < y_\ell < p_i$, then $k <_i \ell$.
  This corresponds to voters having a ``favorite point'' (i.e.
  their peak), and their preference for an alternative decreases
  as you move away from the peak.
  Single peaked preferences are dramatically better behaved
  than general preferences.
  If $P$ is single peaked, then there exist voting schemes
  $f : P^m \to [n]$ which are ``strategy-proof'',
  ``Pareto-optimal'' (a stronger
  version of unanimity), and ``anonymous'' (a stronger version
  of not being a dictatorship) \cite{AgtBookNoMoney}.
  This voting scheme essentially takes the medians of the ``peaks''.

  We introduce the notion of \emph{$d$-dimensional} preferences.
  We say $P$ is $d$-dimensional if there exists a set
  $X = \{x_1,\ldots,x_n\} \subseteq \R_{\ge 0}^d$ such that, for each $>_i \in P$,
  there exists a ``preference weight'' $a_i\in \R_{\ge 0}^d$ such that
  $k >_i \ell$ if and only if $\ip{a_i}{x_k} > \ip{a_i}{x_\ell}$.
  This corresponds to voters choosing options based on a simple weighted
  sum of the attributes of these options.

\section*{$2$-dimensional voting}

  When $d=1$, the $1$-dimensional preference sets are trivial,
  i.e. they contain only one preference.
  The first interesting case, $d=2$, is the main focus and application
  of our project.
  The class of single peaked preferences and $2$-dimensional preferences
  seem vaguely related, simply by looking at what preference lists
  are possible and impossible.
  For example, both preference sets \textsc{Cycle} (a common counterexample
  for certain voting schemes having nice properties) and \textsc{Sandwich},
  shown to the right, are neither single peaked nor $2$-dimensional.
  However, we note that neither class is a subset of the other:
  for example, \textsc{LeastFav} is $2$-dimensional but not single peaked,
  and \textsc{FlipFlop} is single peaked but not two dimensional.
  Finally, \textsc{Reverse} gives a nontrivial preference set which
  is both single peaked and $2$-dimensional.

  Moreover, these one very compelling similarity between $2$-dimensional
  preferences and single-peaked preferences.
  Both correspond to a \emph{one} dimensional set of options:
  the peaks all lie in the one-dimensional interval $[0,1]$,
  and the preference weights can without loss of generality
  all be taken to lie on the \emph{unit circle},
  because $>_a$ is independent of $\|a\|$.
  This gives a natural generalization of the ``median peaks''
  voting algorithm that worked for single peaked preferences.
  Fix some set of ``candidates'' $X\subseteq \Rgz^2$.
  Then our voting scheme is:

  \begin{algorithm}
    Poll a preference weight $a_i\in\Rgz^d$ from each voter $i$ \\
    Select the $a^*$ with median angle (from the $x$ axis) \\
    Return the favorite option of $a^*$.
  \end{algorithm}

  Because this scheme explicity picks some voter's favorite candidate,
  it is Pareto-optimal (and hence unanimous).
  Moreover, if the number of voters is odd, the order of the voters
  does not matter, so the voting scheme is formally called anonymous
  (and hence not a dictatorship).
  \textbf{The main conjecture of our project is that this voting
  scheme is also strategyproof, and satisfies independence
  of irrelevent alternatives}.

  Strategyproofness was easy to prove for the single-peaked
  medians algorith, but the proof does not carry over to our case automatically.
  Strategyproofness is equivalent to a simple combinatorial condidion
  called ``monotonicity''. We have checked for examples
  of non-monotonicity by randomly picking points and preference
  vectors many times, and no examples have turned up.
  Our goal is thus to prove that this rule is strategyproof,
  and if we can, replicate the discussion of \cite{AgtBookNoMoney}
  which fully characterizes strategy-proof single-peaked voting schemes.
  We were not able to test independence of irrelevent alternatives experimentally,
  and that think it may be harder to prove anyway.
  However, we would like to try!

\section*{Higher-dimensional preferences}

  For $d\ge 3$, the situation is less nice.
  Any set of preferences on $3$ options is $3$-dimensional,
  and $3$ options is already sufficient to prove negative
  results such as Arrow's Theorem.
  Thus, good voting schemes for $d > 2$ would require new notions
  of good voting scheme (such as using randomness \cite{AgtBookNoMoney}).
  We do not plan to investigate these.

  However, the proofs that certain preference sets fail to be $d$-dimensional
  are interesting and novel, and we regard the combinatorial classification
  of $d$-dimensional preference sets to be of independent interest.
  So far, we have proofs for all the examples mentioned in the figures,
  and that the generalization of {\sc Cycle} to $d+1$ options
  is impossible in $d$ dimensions.
  The proofs involve considering the convex hulls of the option points,
  and convex cones of the preference weights.
  We do not have a concrete conjecture for a classification of $d$-dimensional
  preferences, but we think that the maximal number of $d$-dimensional
  preferences may be something like $O(n^{d-1})$
  (in particular, we would like to show that the number is linear
  when $d=2$).

  We are attempting to use SMT solvers to automatically check
  if a given list is $2$-dimensional, and we would ideally like
  an efficient algorithm for finding the minimal dimension $d$
  needed such that a set of preferences is $d$-dimensional.
  Moreover, we plan to consider how voting schemes should work,
  if voters are know to be $2$ dimensional, but the underlying space
  $X$ is unknown and voters must submit a ranked preference list $<_a$
  (instead of their preference weight vector $a$).
  We believe we can adapt our medians algorithm to work in this case.

  \bibliography{weightedPref}{}
  \bibliographystyle{alpha}

\end{document}
