\documentclass[12pt]{article}
 
\usepackage[noend]{algpseudocode}
\usepackage{algorithm}
\usepackage{algorithmicx}
\usepackage{float}
\usepackage{graphicx}
\usepackage[margin=.75in]{geometry} 
\usepackage{amsmath,amsthm,amssymb}
\usepackage{dsfont}
\usepackage{subcaption}
\usepackage{amsthm}
\usepackage{mathtools,amssymb}
\usepackage{wrapfig}
\usepackage[font=small,skip=-20pt]{caption}
\allowdisplaybreaks

\newtheorem*{definition}{Definition}
\newtheorem*{question}{Question}
\newtheorem{theorem}{Theorem}
\newtheorem{proposition}[theorem]{Proposition}
\newtheorem{claim}[theorem]{Claim}
\newtheorem{lemma}[theorem]{Lemma}
\newtheorem{corollary}[theorem]{Corollary}
\newtheorem{conjecture}[theorem]{Conjecture}
 
\newcommand{\N}{\mathbb{N}}
\newcommand{\Z}{\mathbb{Z}}
\newcommand{\R}{\mathbb{R}}
\newcommand{\Rgz}{\mathbb{R}_{\ge 0}}

\newcommand{\ip}[2]{\left\langle{#1},{#2}\right\rangle}

\newcommand{\woloss}{without loss of generality }

\DeclareMathOperator*{\argmin}{arg\,min}
\DeclareMathOperator*{\argmax}{arg\,max}
\DeclareMathOperator*{\cone}{cone}
\DeclareMathOperator*{\hull}{hull}

\newcommand{\1}[1]{\mathds{1}[{#1}]}
\renewcommand{\P}[1]{\mathds{P}\left[{#1}\right]}
\newcommand{\E}[1]{\mathds{E}\left[{#1}\right]}
\newcommand{\Var}[1]{\mathrm{Var}[{#1}]}

\newcommand{\unit}{\mathds{1}}
\newcommand{\lo}{\succ}

% \renewcommand{\thesubsection}{\thesection.\alph{subsection}}
% \renewcommand\thesubsection{\ \ (\alph{subsection})}


\begin{document}

% \renewcommand{\qedsymbol}{\filledbox}
 
\title{Voting with $2$-dimensional preferences}
\author{
  Clay Thomas \\
  claytont@princeton.edu
\and
  Yufei Zheng\\
  yufei@cs.princeton.edu
}

\maketitle
\begin{abstract}
  The study of voting rules often restricts attention to
  well-behaved classes of possible preferences of voters.
  We define a new class: $2$-dimensional preferences,
  for which very good voting rules are possible.
  We argue that $2$-dimensional preferences are in some ways
  more natural and expressive than more traditional classes
  such as single-peaked preferences.
  Furthermore, we give an almost-complete combinatorial classification
  of $2$-dimensional preferences, and provide some additional
  results about the natural extension of $d$-dimensional preferences.
\end{abstract}

\section{Introduction}






%============================================================
%============================================================
\part{Voting}

\section{Definitions}
  \subsection{Voting}
    Consider a set of preferences $P$ (i.e. linear orders) on a
    set of outcomes $M$.
    We follow the convention that $m$ is the number of outcomes and $n$
    is the number of voters.
    We have the following definitions:
    \begin{itemize}
      \item A \emph{social welfare function on $P$}
        is a function $F : P^n \to P$.
      \item A \emph{social choice function on $P$} 
        is a function $f : P^n \to M$.
      \item A welfare function $F$ is \emph{unanimous} if,
        for any $\succ \in P$, we have $F(\succ,\ldots,\succ) = \succ$.
      \item A choice function $f$ is \emph{unanimous} if,
        whenever there exists a fixed $a$ with $a \succ_i b$ for all 
        $i=1,\ldots, n$ and $b\in M\setminus \{a\}$,
        we have $f(\succ_1,\ldots,\succ_n) = a$.
        I'm not sure if this is a standard notion.
      \item A welfare function $F$ is a \emph{dictatorship} if
        there exists an $i$ such that $F(\succ_1,\ldots,\succ_n) = \succ_i$.
      \item A welfare function $F$ satisfies \emph{in dependence of
        irrelevant alternatives} if, whenever $a\succ_i b \iff a\succ_i' b$
        and $\succ = F(\succ_1,\ldots,\succ_n),
        \succ' = F(\succ'_1,\ldots,\succ'_n)$,
        we get $a\succ b \iff a\succ' b$
      \item A choice function $f$ is \emph{incentive compatible} if,
        for any $\succ_1,\ldots,\succ_n, i, \succ_i'$, we have \\
        $f(\succ_1,\ldots,\succ_n) \succ_i f(\succ_1,\ldots,\succ_i'\ldots,\succ_n)$
    \end{itemize}

    Let $R(M)$ denote the set of all linear orders on $M$.
    Recall that when $P = R(M)$, there are many known impossibility results,
    the two most famous of which are:

    ((Arrow's))

    ((G-S))

  \subsection{Single-peaked preferences}
    Our main point of contrast will be the well-understood class of
    \emph{single peaked} preferences.

    Let $S\subseteq [0,1]$ be a finite set of $m$ points in the unit interval.
    We call $S$ the set of \emph{outcomes}.
    Let $R(S)$ denote the set of linear orders on $X$.
    A preference $\succ \in R(S)$ is called \emph{single peaked} if
    there exists an outcome $p\in S$ (called the ``peak'' of $\succ$)
    such that $x < y < p \implies x \prec y$ and $p < y < x \implies x \prec y$.
    In other words, the preference has a favorite outcome,
    and its opinion strictly decreases as you move farther away from the favorite.
    Note that no assumption is made about outcomes on different sides of the peak.
    Define
    \begin{align*}
      P_{sp}(S) = \{ \succ \in R(S) | \succ \text{is single peaked }\}
    \end{align*}


  \subsection{$d$-dimensional preferences}
    Let $\{x_1,\ldots,x_m\} = X\subseteq \Rgz^d$ be any (ordered) set of
    $m$ distinct points with nonnegative coordinates.
    We call $X$ the set of \emph{outcomes}.
    Given any $a\in \Rgz^d$, define a order $>_a$ on $X$ as follows:
    $x_i >_a x_j$ if and only if $\ip{a}{x_i} > \ip{a}{x_j}$.
    Define
    \begin{align*}
      P_d(X) = \{ \succ \in R(X) | \exists a\in\Rgz^d: x \succ y \iff x >_a y\}
    \end{align*}

\section{Single-peaked verses $2$-dimensional preferences}

  When there are exactly three candidates, $2$-dimensional
  preferences have strictly more expressive power than
  single-peaked preferences.
  \begin{proposition}
    Every single-peaked preference set on $m=3$ outcomes
    is a $2$-dimensional preference set. In particular,
    up to relabeling it is a subset of $P_2(X)$, for
    $X = \{ (3,0), (2,2), (0,3) \}$.
  \end{proposition}
  \begin{proof}
    Without loss of generality, relabel the outcomes
    of the single peaked preferences as $S = \{1,2,3\}$.
    The resulting preferences are given by {\sc GoodCompromise}.
    ((INSERT PICTURE OF GOOD COMPROMISE AS A 2D SET))
  \end{proof}
  Furthermore, {\sc LeastFavorite} gives an example of a preference set
  which is $2$-dimensional, but not single-peaked:
  \begin{proposition}
    In a single peaked set of preferences, it is not possible
    for every outcome to be the lowest ranked outcome of some preference.
  \end{proposition}
  \begin{proof}
    The median outcome (via the standard order on $[0,1]$)
    cannot be the lowest ranked.
  \end{proof}

  When $m>3$, neither single-peaked nor $2$-dimensional preferences
  are a subset of the other. One was to see this to to simply
  count the number of single-peaked preferences, and see that it
  grows much faster with $m$ than $2$-dimensional preferences do
  (we'll see later, the maximal size of a $2$-dimensional
  preference set is $O(m^2)$).
  \begin{proposition}
    The number of single-peaked preferences for any set $S$ of $m$ outcomes
    is at least $2^{\Omega(m)}$
  \end{proposition}
  \begin{proof}
    Choose the median outcome of $S$ to be the peak.
    Given a subset $T\subseteq [m-1]$ of size $|T| = \lfloor m/2\rfloor$,
    we can associate a
    unique single-peaked preference as follows:
    treat the preference as an array of outcomes, ranked highest to lowest.
    Put the median of $S$ at index $0$.
    Let the outcomes to the left of the median occupy the indices corresponding
    to set $T$, and let those to the right occupy the other indices.
    There are ${m-1 \choose \lfloor m/2\rfloor } \ge 2^{\Omega(m)}$ such
    subsets $T$.
  \end{proof}

  A cleaner, more constructive way to see this is the following proposition
  \begin{proposition}
    The preference set {\sc FlipFlop} is single-peaked, but not
    $2$-dimensional.
  \end{proposition}
  \begin{proof}
    To obtain a single-peaked representation, order the outcomes
    from left to right as follows: $3,1,2,4$.
    All preferences will then be possible with peak $1$ or $2$.

    ((INSERT IMPOSSIBILITY HALF))
  \end{proof}

  ((DISCUSS SETS WHICH ARE NEITHER SINGLE-PEAKED NOR 2-D))

  ((ARGUE THAT 2-D IS MORE NATURAL THAN SINGLE-PEAKED,
  EVEN THOUGH THERE ARE (ASYMPTOTICALLY) MORE SINGLE-PEAKED))

\clearpage

  % TODO: at end, put this at start of whichever page makes the most sense
  \begin{wrapfigure}{tr}{0.2\textwidth}
    \begin{center}
      \vspace{-0.5in}
      \begin{align*}
        1 > 2 > 3 \\
        2 > 3 > 1 \\
        3 > 1 > 2 \\
      \end{align*}
      \caption*{\textsc{Cycle}}
      \vspace{-0.25in}
      \begin{align*}
        1 > 2 > 3 \\
        1 > 3 > 2 \\
        3 > 2 > 1 \\
        2 > 3 > 1 \\
      \end{align*}
      \caption*{\textsc{Sandwich}}
      \vspace{-0.25in}
      \begin{align*}
        1 > 2 > 3 \\
        1 > 3 > 2 \\
        3 > 2 > 1 \\
      \end{align*}
      \caption*{\textsc{LeastFavorite}}
      \vspace{-0.25in}
      \begin{align*}
        1 > 2 > 3 > 4 \\
        1 > 2 > 4 > 3 \\
        2 > 1 > 3 > 4 \\
        2 > 1 > 4 > 3 \\
      \end{align*}
      \caption*{\textsc{FlipFlop}}
      \vspace{-0.25in}
      \begin{align*}
        1 > 2 > 3 > 4 \\
        4 > 3 > 2 > 1 \\
      \end{align*}
      \caption*{\textsc{Reverse}}
      \vspace{-0.25in}
      \begin{align*}
        1 > 2 > 3 \\
        2 > 1 > 3 \\
        2 > 3 > 1 \\
        3 > 2 > 1 \\
      \end{align*}
      \caption*{\textsc{GoodCompromise}}
    \end{center}
  \end{wrapfigure}

\section{Voting for $2$-dimensional preferences}

\section{$2$-dim entails a Condorcet winner}

%============================================================
%============================================================
\part{Combinatorics}

\section{Bounding the number of $2$-dimensional preferences}

\end{document}
