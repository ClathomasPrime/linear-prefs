\documentclass[12pt]{article}
 
\usepackage[noend]{algpseudocode}
\usepackage{algorithm}
\usepackage{algorithmicx}
\usepackage{float}
\usepackage{graphicx}
\usepackage[margin=.75in]{geometry} 
\usepackage{amsmath,amsthm,amssymb}
\usepackage{dsfont}
\usepackage{subcaption}
\usepackage{amsthm}
\usepackage{mathtools,amssymb}
\usepackage{wrapfig}
\usepackage[font=small,skip=-20pt]{caption}
\allowdisplaybreaks

\newtheorem*{definition}{Definition}
\newtheorem*{question}{Question}
\newtheorem{theorem}{Theorem}
\newtheorem{proposition}[theorem]{Proposition}
\newtheorem{claim}[theorem]{Claim}
\newtheorem{lemma}[theorem]{Lemma}
\newtheorem{corollary}[theorem]{Corollary}
\newtheorem{conjecture}[theorem]{Conjecture}
 
\newcommand{\N}{\mathbb{N}}
\newcommand{\Z}{\mathbb{Z}}
\newcommand{\R}{\mathbb{R}}
\newcommand{\Rgz}{\mathbb{R}_{\ge 0}}

\newcommand{\ip}[2]{\left\langle{#1},{#2}\right\rangle}

\newcommand{\woloss}{without loss of generality }

\DeclareMathOperator*{\argmin}{arg\,min}
\DeclareMathOperator*{\argmax}{arg\,max}
\DeclareMathOperator*{\cone}{cone}
\DeclareMathOperator*{\hull}{hull}

\newcommand{\1}[1]{\mathds{1}[{#1}]}
\renewcommand{\P}[1]{\mathds{P}\left[{#1}\right]}
\newcommand{\E}[1]{\mathds{E}\left[{#1}\right]}
\newcommand{\Var}[1]{\mathrm{Var}[{#1}]}

\newcommand{\unit}{\mathds{1}}
\newcommand{\lo}{\succ}

% \renewcommand{\thesubsection}{\thesection.\alph{subsection}}
% \renewcommand\thesubsection{\ \ (\alph{subsection})}


\begin{document}

% \renewcommand{\qedsymbol}{\filledbox}
 
\title{Voting with $2$-dimensional preferences}
\author{
  Clay Thomas \\
  claytont@princeton.edu
\and
  Yufei Zheng\\
  yufei@cs.princeton.edu
}

\maketitle
\begin{abstract}
  The study of voting rules often restricts attention to
  well-behaved classes of possible preferences of voters.
  We define a new class: $2$-dimensional preferences,
  for which very good voting rules are possible.
  We argue that $2$-dimensional preferences are in some ways
  more natural and expressive than more traditional classes
  such as single-peaked preferences.
  Furthermore, we give an almost-complete combinatorial classification
  of $2$-dimensional preferences, and provide some additional
  results about the natural extension of $d$-dimensional preferences.
\end{abstract}

\section{Introduction}

  We believe that $d$-dimensional preferences are a very natural setting,
  and that they give a beautiful geometric explanation for psychological
  concepts such as opinion and agreement. 



\clearpage
%============================================================
%============================================================
\part{Combinatorics}

\section{Definition of $d$-dimensional preferences}
  Let $\{x_1,\ldots,x_m\} = X\subseteq \Rgz^d$ be any (ordered) set of
  $m$ distinct points with nonnegative coordinates.
  We call $X$ the set of \emph{outcomes}.
  Given any $a\in \Rgz^d$, define a order $>_a$ on $X$ as follows:
  $x_i >_a x_j$ if and only if $\ip{a}{x_i} > \ip{a}{x_j}$.
  Define
  \begin{align*}
    P_d(X) = \{ \succ \in R(X) | \exists a\in\Rgz^d: x \succ y \iff x >_a y\}
  \end{align*}

  Some first observations:
  \begin{itemize}
    \item If $d=1$, then $|P(X)| = 1$, i.e. preferences are completely
      determined by the underlying set $X$.
    \item If $d=n$, then every linear preference on $X$ can occur in $P(X)$.
      \begin{proof}
        Let $X = \{e_i\}$ simply be the standard basis vectors.
        To induce an ordering $i_1, i_2, \ldots, i_n$, just create a preference
        vector $a$ which gives weight $1/k$ to coordinate $i_k$.
      \end{proof}
  \end{itemize}
  The above hints that there should be some sort of continuum between $d=1$
  and $d=n$ of how complex the set $P(X)$ can be.

\section{Lemmas}
  \subsection{Lemmas based on the structure of $X$}
    These first few lemmas relate geometric properties of $X$ to
    limitations on the structure of $>_a$ for a specific, fixed $a$.

    \begin{definition}
      Let $x,y\in X\subseteq \Rgz^d$.
      We say $x$ \emph{dominates} $y$, denoted $x\gg y$,
      if $x[k] > y[k]$ for each $k=1,\ldots,d$.
    \end{definition}
    \begin{proposition}
      If $x \gg y$, then $x >_a y$ for any nonzero $a\in \Rgz^d$.
    \end{proposition}
    \begin{proof}
      Simply observe $a_i x_i \ge a_i y_i$ for each $i=1,\ldots, d$.
      Because $a\ne 0$, there is also some $i$ where $a_i x_i > a_i y_i$.
    \end{proof}

    We'll need to use the familiar geometric notion of a convex hull:
    ((ACTUALLY, I'M NOT SURE IF WE DO))
    \begin{definition}
      For points $x_1,\ldots,x_n \in \R^d$, let $\hull(x_1,\ldots,x_n)
      = \{u_1x_1 + \ldots + u_nx_n | 0\le u_i\le 1, \sum_{i=1}^n u_i = 1\}$
      denote the convex hull of $x_1,\ldots,x_n$.
    \end{definition}
    \begin{lemma}
      Let $z,x_1,\ldots,x_k \in \Rgz^d$ and $a\in \Rgz^d \setminus \{0\}$.
      If $z >_a x_i$ for $i=1,\ldots,k$, then $z >_a w$
      for any $w\in \hull(x_1,\ldots,x_k)$.
    \end{lemma}
    \begin{proof}
      We have $\ip{a}{z} > \ip{a}{x_i}$ for each $i=1,\ldots,k$.
      If $w = u_1x_1+ \ldots + u_nx_n$ and $\sum_i u_i =1$, then
      $\ip{a}{w} = u_1\ip{a}{x_1}+\ldots+u_n\ip{a}{x_n}
      < u_1\ip{a}{z} + \ldots + u_n\ip{a}{z} = \ip{a}{z}$
    \end{proof}
    \begin{proposition}
      Let $z,x_1,\ldots,x_k \in \Rgz^d$.
      Suppose that there exists $w\in \hull(x_1,\ldots,x_k)$
      such that $w \gg z$.
      Then no $a$ satisfies $z >_a x_i$ for each $i=1,\ldots, k$.
    \end{proposition}
    \begin{proof}
      For contradiction, suppose such an $a$ exists.
      Then $z >_a w$ as well. However, because $w \gg z$,
      this is a contradiction.
    \end{proof}

  \subsection{Lemmas relating different preference vectors}
    This next group of lemmas relates different 

    \begin{definition}
      For any vectors $a_1,\ldots, a_k \in \Rgz^d$, define
      $\cone(a_1,\ldots,a_k) = \{ u_1a_1 + \ldots + u_ka_k | u_i\ge 0 \forall j\}$.
      That is, the convex cone of vectors defined by $a_1,\ldots,a_k$.
    \end{definition}
    \begin{proposition}
      Suppose that for preference weights $a_1,\ldots, a_k$,
      we have $x >_{a_1} y, \ldots, x >_{a_k} y$.
      Then for any nonzero $b\in \cone(a_1,\ldots, a_k)$,
      $x >_b y$ as well.
    \end{proposition}
    \begin{proof}
      Let $b = u_1a_1+ \ldots + u_ka_k$.
      We get $\ip{b}{x} = u_1\ip{a_1}{x} + \ldots + u_k\ip{a_k}{x}
      > u_1\ip{a_1}{y} + \ldots + u_k\ip{a_k}{y} = \ip{b}{y}$,
      as desired.
    \end{proof}

    ((INSERT LEMMAS ABOUT SLOPES))

\section{Bounding the number of $2$-dimensional preferences}

\section{Impossibility Results}

  ((CYCLE IS IMPOSSIBLE IN ANY DIMENSION))

  ((MOST AND LEAST FAVORITE IS IMPOSSIBLE FOR D=2,
  mention that later we'll apply this for Sandwich))

  ((SET THE STAGE TO PROVE FLIPFLOP IS IMPOSSIBLE FOR D=2))


\clearpage
%============================================================
%============================================================
\part{Voting}

\section{Definitions}
  \subsection{Voting}
    Consider a set of preferences $P$ (i.e. linear orders) on a
    set of outcomes $M$.
    We follow the convention that $m$ is the number of outcomes and $n$
    is the number of voters.
    We have the following definitions:
    \begin{itemize}
      \item ((NOTE: we don't need most of these definitions))
      \item A \emph{social welfare function on $P$}
        is a function $F : P^n \to P$.
      \item A \emph{social choice function on $P$} 
        is a function $f : P^n \to M$.
      \item A welfare function $F$ is \emph{unanimous} if,
        for any $\succ \in P$, we have $F(\succ,\ldots,\succ) = \succ$.
      \item A choice function $f$ is \emph{unanimous} if,
        whenever there exists a fixed $a$ with $a \succ_i b$ for all 
        $i=1,\ldots, n$ and $b\in M\setminus \{a\}$,
        we have $f(\succ_1,\ldots,\succ_n) = a$.
        I'm not sure if this is a standard notion.
      \item A welfare function $F$ is a \emph{dictatorship} if
        there exists an $i$ such that $F(\succ_1,\ldots,\succ_n) = \succ_i$.
      \item A welfare function $F$ satisfies \emph{in dependence of
        irrelevant alternatives} if, whenever $a\succ_i b \iff a\succ_i' b$
        and $\succ = F(\succ_1,\ldots,\succ_n),
        \succ' = F(\succ'_1,\ldots,\succ'_n)$,
        we get $a\succ b \iff a\succ' b$
      \item A choice function $f$ is \emph{incentive compatible} if,
        for any $\succ_1,\ldots,\succ_n, i, \succ_i'$, we have \\
        $f(\succ_1,\ldots,\succ_n) \succ_i f(\succ_1,\ldots,\succ_i'\ldots,\succ_n)$
      \item A collection of preferences $\succ_1,\ldots,\succ_n$,
        has a Condorcet winner $x \in M$ if, for any other $y\in M$,
        we have $x \succ_i y$ for more than half of the indices
        $i=1,\ldots,n$.
    \end{itemize}

    Let $R(M)$ denote the set of all linear orders on $M$.
    Recall that when $P = R(M)$, there are many known impossibility results,
    the two most famous of which are:

    ((Arrow's))

    ((G-S))

  \subsection{Single-peaked preferences}
    Our main point of contrast will be the well-understood class of
    \emph{single peaked} preferences.

    Let $S\subseteq [0,1]$ be a finite set of $m$ points in the unit interval.
    We call $S$ the set of \emph{outcomes}.
    Let $R(S)$ denote the set of linear orders on $X$.
    A preference $\succ \in R(S)$ is called \emph{single peaked} if
    there exists an outcome $p\in S$ (called the ``peak'' of $\succ$)
    such that $x < y < p \implies x \prec y$ and $p < y < x \implies x \prec y$.
    In other words, the preference has a favorite outcome,
    and its opinion strictly decreases as you move farther away from the favorite.
    Note that no assumption is made about outcomes on different sides of the peak.
    Define
    \begin{align*}
      P_{sp}(S) = \{ \succ \in R(S) | \succ \text{is single peaked }\}
    \end{align*}


\section{Single-peaked verses $2$-dimensional preferences}

  % TODO: at end, put this at start of whichever page makes the most sense
  \begin{wrapfigure}{r}{0.2\textwidth}
    \begin{center}
      \vspace{-0.5in}
      \begin{align*}
        1 > 2 > 3 \\
        2 > 3 > 1 \\
        3 > 1 > 2 \\
      \end{align*}
      \caption*{\textsc{Cycle}}
      \vspace{-0.25in}
      \begin{align*}
        1 > 2 > 3 \\
        1 > 3 > 2 \\
        3 > 2 > 1 \\
        2 > 3 > 1 \\
      \end{align*}
      \caption*{\textsc{Sandwich}}
      \vspace{-0.25in}
      \begin{align*}
        1 > 2 > 3 \\
        1 > 3 > 2 \\
        3 > 2 > 1 \\
      \end{align*}
      \caption*{\textsc{LeastFavorite}}
      \vspace{-0.25in}
      \begin{align*}
        1 > 2 > 3 > 4 \\
        1 > 2 > 4 > 3 \\
        2 > 1 > 3 > 4 \\
        2 > 1 > 4 > 3 \\
      \end{align*}
      \caption*{\textsc{FlipFlop}}
      \vspace{-0.25in}
      \begin{align*}
        1 > 2 > 3 > 4 \\
        4 > 3 > 2 > 1 \\
      \end{align*}
      \caption*{\textsc{Reverse}}
      \vspace{-0.25in}
      \begin{align*}
        1 > 2 > 3 \\
        2 > 1 > 3 \\
        2 > 3 > 1 \\
        3 > 2 > 1 \\
      \end{align*}
      \caption*{\textsc{GoodCompromise}}
    \end{center}
  \end{wrapfigure}


  When there are exactly three candidates, $2$-dimensional
  preferences have strictly more expressive power than
  single-peaked preferences.
  \begin{proposition}
    Every single-peaked preference set on $m=3$ outcomes
    is a $2$-dimensional preference set. In particular,
    up to relabeling it is a subset of $P_2(X)$, for
    $X = \{ (3,0), (2,2), (0,3) \}$.
  \end{proposition}
  \begin{proof}
    Without loss of generality, relabel the outcomes
    of the single peaked preferences as $S = \{1,2,3\}$.
    The resulting preferences are given by {\sc GoodCompromise}.
    ((INSERT PICTURE OF GOOD COMPROMISE AS A 2D SET))
  \end{proof}
  Furthermore, {\sc LeastFavorite} gives an example of a preference set
  which is $2$-dimensional, but not single-peaked:
  \begin{proposition}
    In a single peaked set of preferences, it is not possible
    for every outcome to be the lowest ranked outcome of some preference.
  \end{proposition}
  \begin{proof}
    The median outcome (via the standard order on $[0,1]$)
    cannot be the lowest ranked.
  \end{proof}

  When $m>3$, neither single-peaked nor $2$-dimensional preferences
  are a subset of the other. One was to see this to to simply
  count the number of single-peaked preferences, and see that it
  grows much faster with $m$ than $2$-dimensional preferences do
  (we'll see later, the maximal size of a $2$-dimensional
  preference set is $O(m^2)$).
  \begin{proposition}
    The number of single-peaked preferences for any set $S$ of $m$ outcomes
    is at least $2^{\Omega(m)}$
  \end{proposition}
  \begin{proof}
    Choose the median outcome of $S$ to be the peak.
    Given a subset $T\subseteq [m-1]$ of size $|T| = \lfloor m/2\rfloor$,
    we can associate a
    unique single-peaked preference as follows:
    treat the preference as an array of outcomes, ranked highest to lowest.
    Put the median of $S$ at index $0$.
    Let the outcomes to the left of the median occupy the indices corresponding
    to set $T$, and let those to the right occupy the other indices.
    There are ${m-1 \choose \lfloor m/2\rfloor } \ge 2^{\Omega(m)}$ such
    subsets $T$.
  \end{proof}

  A cleaner, more constructive way to see this is the following proposition
  \begin{proposition}
    The preference set {\sc FlipFlop} is single-peaked, but not
    $2$-dimensional.
  \end{proposition}
  \begin{proof}
    To obtain a single-peaked representation, order the outcomes
    from left to right as follows: $3,1,2,4$.
    All preferences will then be possible with peak $1$ or $2$.
    ((INSERT IMPOSSIBILITY HALF))
  \end{proof}

  ((DISCUSS SETS WHICH ARE NEITHER SINGLE-PEAKED NOR 2-D))

  ((ARGUE THAT 2-D IS MORE NATURAL THAN SINGLE-PEAKED,
  EVEN THOUGH THERE ARE (ASYMPTOTICALLY) MORE SINGLE-PEAKED))
\section{Voting for $2$-dimensional preferences}

  \begin{theorem}
    The median-angle voting rule for $2$-dimensional preferences
    is incentive compatible.
  \end{theorem}
  \begin{proof}
    Let $u$ denote the median-angled preference vector under when all
    agents report their true preference.
    Suppose the manipulative voter $i$'s true preference $a$
    has angle less than $u$ (the other case is symmetric).
    The only way in which $i$ can change the chosen preference vector
    is to increase its angle, say to some $v$.
    Note that $u \in \cone(v,a)$.

    Suppose $y$ is the outcome chosen by the non-manipulated median
    angle preference $u$.
    Now, in order for $i$ to profit by his manipulation,
    there must be some $x$ such that
    $x >_a y$, $y >_u x$, and $x >_v y$.
    However, this is a contradiction, via the lemma about cones
    preserving agreed upon opinions.

    ((PROOF NEEDS REVISION))
  \end{proof}

\section{$2$-dimensional entails a Condorcet winner}

  \begin{theorem}
    Among an odd number of $2$-dimensional preferences,
    there is always a Condorcet winner, which is selected
    by the median angle voting scheme
  \end{theorem}
  \begin{proof}
    Let $u$ be the median-angled preference vector, and let
    $x$ be the favorite outcome of $u$.
    Given some outcome $y\ne x$, if $y \ll x$, then
    $y$ is never preferred to $x$.
    So suppose neither $x$ nor $y$ dominate each other, and let
    $w$ be the vector such that $x =_w y$.
    On one side of $w$, preferences prefer $x$ to $y$,
    and on the other side, they prefer $y$ to $x$.
    Because $x$ is preferred by the median-angled voter,
    more than half of the preference vectors lie on the side of $w$
    which prefers $x$.

    ((PROOF NEEDS REVISION))
  \end{proof}

  We note the following important consequence of the previous theorem:
  if preferences are known to lie in $P_2(X)$ for some $X$,
  but no details about $X$ are known, then a voting protocol can simply
  ask voters for their list of preferences and look for the Condorcet
  winner (note that there is at most one Condorcet winner).
  Moreover, it hints that when agents act approximately via a weighted
  sum of two attributes, it is more likely that there is a Condorcet winner.

\end{document}
