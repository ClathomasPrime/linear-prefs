\documentclass[12pt]{article}

\usepackage{comment}
 
\usepackage[noend]{algpseudocode}
\usepackage{algorithm}
\usepackage{float}
\usepackage{graphicx}
\usepackage[margin=.75in]{geometry} 
\usepackage{amsmath,amsthm,amssymb}
\usepackage{amsthm}
\usepackage{mathtools,amssymb}
\usepackage{filecontents}
\usepackage{hyperref}

\newtheorem{theorem}{Theorem}
\newtheorem{definition}[theorem]{Definition}
\newtheorem{question}[theorem]{Question}
\newtheorem{example}[theorem]{Example}
\newtheorem{proposition}[theorem]{Proposition}
\newtheorem{claim}[theorem]{Claim}
\newtheorem{lemma}[theorem]{Lemma}
\newtheorem{corollary}[theorem]{Corollary}
\newtheorem{conjecture}[theorem]{Conjecture}

\newcommand{\N}{\mathbb{N}}
\newcommand{\Z}{\mathbb{Z}}
\newcommand{\R}{\mathbb{R}}
\newcommand{\C}{\mathcal{C}}
\newcommand{\D}{\mathcal{D}}

\newcommand{\sizeof}[1]{\left\lvert{#1}\right\rvert}

\DeclareMathOperator*{\argmin}{arg\,min}
\DeclareMathOperator*{\argmax}{arg\,max}
\DeclareMathOperator*{\rev}{rev}

\newcommand{\1}[1]{\mathds{1}[{#1}]}
\renewcommand{\P}[1]{\mathds{P}\left[{#1}\right]}
\newcommand{\E}[1]{\mathds{E}\left[{#1}\right]}
\newcommand{\Var}[1]{\mathrm{Var}[{#1}]}

\newcommand{\unit}{\mathds{1}}
\newcommand{\lo}{\succ}

\begin{document}

\title{
  Condorcet Domains
}
\author{
  Clay Thomas \\
  claytont@cs.princeton.edu
\and
  Corey Sinnamon \\
  sinnamon@cs.princeton.edu
}

\maketitle

\section{Introduction}

  Arrow's impossibility theorem says voting doesn't work out like you want it to.
  Condorcet domains work around this.

  \begin{theorem} \label{thrmCondorcetProps}
    Let $\C \subseteq S_n$ be some set of linear orders.
    The following are equivalent:
    \begin{enumerate}
      \item Simple majority voting over any profile of voters is always acyclic.
      \item Simple majority voting over any odd-size profile of voters
        always yields a total order.
      \item Every triple of outcomes $i,j,k\in[n]$ is \emph{acyclic}, meaning
        that $\C$ restricted to $\{i,j,k\}$ does not contain the orders
        $ijk$, $jki$, and $kij$.
      \item \label{thrmCondorcetPropsValRestr}
        Among ever triple $i,j,k\in [n]$ of distinct outcomes in $[n]$,
        one of them is either never first, never last, or never in the middle.
    \end{enumerate}

  \end{theorem}
  \begin{definition}
    A Condorcet domain $\C\subseteq S_n$ is any set of orders satisfying any of
    the above conditions.
  \end{definition}

\section{Preliminaries}

  If $\C$ is a Condorcet domain, theorem~\ref{thrmCondorcetProps}
  part~\ref{thrmCondorcetPropsValRestr} tells us that every triple of outcomes
  falls into some ``value restriction case'': one of $i$, $j$, or $k$ is either
  never first, never second, or never third among $\{i,j,k\}$.
  In those cases, if $u$ is the ``restricted'' outcome among $\{i,j,k\}$,
  we write $u N_1$, $u N_2$, or $u N_3$, respectively.
  If the triple $\{i,j,k\}$ is not clear from the context, we will write
  $u N_i vw$ for $\{v,w\} = \{i,j,k\} \setminus \{u\}$.

  \begin{definition}
    A \emph{value restriction (VR) casting} on $[n]$ is a function
    $\psi$ from triples of outcomes $(i,j,k) \in {[n] \choose 3}$
    to one of the nine ``value restriction cases''
    $u N_v$ for $u\in\{i,j,k\}$ and $v\in \{1,2,3\}$.

    The corresponding domain $\D(\psi)$ of some value restriction 
    is the collection of all orders $\sigma \in S_n$ which satisfy $\psi$ on
    every triple. That is, $\sigma\in\D(\psi)$ if for all $i,j,k \in [n]$ with
    $\psi(i,j,k) = u N_v$, $\sigma|_{i,j,k}$ does not have $u$ in position $v$.
  \end{definition}

  CDs, even closed CDs, are not in general defined
  by the VR castings they satisfy\footnote{
    For example, single crossing domains $\C$ are closed but 
    in general much smaller than $\D(\psi(\C))$.
  }. However, maximal CDs are always the corresponding domain
  of a VR casting:

  \begin{proposition}
    If $\C$ is a maximal CD, then $\C = \D(\psi)$ for some $\psi$.
  \end{proposition}
  \begin{proof}
    By theorem~\ref{thrmCondorcetProps} part~\ref{thrmCondorcetPropsValRestr},
    every CD is contained in $\D(\psi)$ for some $\psi$.
    If $\C\subsetneq \D(\psi)$, then $\C$ is not maximal.
  \end{proof}

  \begin{definition}
    We distinguish the two natural orders
    $\alpha = 12\ldots n$ and $\omega = n(n-1)\ldots 1$.

    The reversed order of any $\sigma\in S_n$
    is given by $\rev(\sigma) = \sigma(n)\sigma(n-1)\ldots\sigma(1)$

    A domain $\C$ is \emph{normal} if $\alpha\in\C$ and $\omega\in\C$.
  \end{definition}

\section{Peak/Pit Domains}

  VR castings are especially useful when studying restricted classes of
  Condorcet domains. In special cases, some simple observations about which VR
  cases are possible allow us to reduce the complexity of the set of VR castings
  considerably.

\section{Bad News -- Counterexamples}

  The property of maximality gives a lot of structure to Condorcet domains.
  However, this structure is complicated.

  \begin{example}
    There are maximal Condorcet domains without full projection onto some
    triple. Moreover, maximality is not preserved under restrictions.
  \end{example}

\end{document}

