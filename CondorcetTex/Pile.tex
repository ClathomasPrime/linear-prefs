\documentclass[12pt]{article}

\usepackage{comment}
 
\usepackage[noend]{algpseudocode}
\usepackage{algorithm}
\usepackage{float}
\usepackage{graphicx}
\usepackage[margin=.75in]{geometry} 
\usepackage{amsmath,amsthm,amssymb}
\usepackage{amsthm}
\usepackage{mathtools,amssymb}
\usepackage{filecontents}
\usepackage{hyperref}

\newtheorem{theorem}{Theorem}
\newtheorem{definition}[theorem]{Definition}
\newtheorem{question}[theorem]{Question}
\newtheorem{example}[theorem]{Example}
\newtheorem{proposition}[theorem]{Proposition}
\newtheorem{claim}[theorem]{Claim}
\newtheorem{lemma}[theorem]{Lemma}
\newtheorem{corollary}[theorem]{Corollary}
\newtheorem{conjecture}[theorem]{Conjecture}

\newcommand{\N}{\mathbb{N}}
\newcommand{\Z}{\mathbb{Z}}
\newcommand{\R}{\mathbb{R}}
\newcommand{\C}{\mathcal{C}}
\newcommand{\D}{\mathcal{D}}
\newcommand{\PP}{\mathcal{P}}
\newcommand{\PowerSet}[1]{\mathbf{2}^{#1}}

\newcommand{\sizeof}[1]{\left\lvert{#1}\right\rvert}

\DeclareMathOperator*{\argmin}{arg\,min}
\DeclareMathOperator*{\argmax}{arg\,max}
\DeclareMathOperator*{\rev}{rev}
\DeclareMathOperator*{\inv}{inv}
\DeclareMathOperator*{\med}{med}
\DeclareMathOperator*{\topChoice}{top}
\DeclareMathOperator*{\last}{last}

\newcommand{\1}[1]{\mathds{1}[{#1}]}
\renewcommand{\P}[1]{\mathds{P}\left[{#1}\right]}
\newcommand{\E}[1]{\mathds{E}\left[{#1}\right]}
\newcommand{\Var}[1]{\mathrm{Var}[{#1}]}

\newcommand{\unit}{\mathds{1}}
\newcommand{\lo}{\succ}

\begin{document}

\title{
  Condorcet Domains
}
\author{
  Clay Thomas \\
  claytont@cs.princeton.edu
\and
  Corey Sinnamon \\
  sinnamon@cs.princeton.edu
}

\maketitle

\section{Introduction}

  Arrow's impossibility theorem says voting doesn't work out like you want it to.
  Condorcet domains work around this.

  \begin{theorem} \label{thrmCondorcetProps}
    Let $\C \subseteq S_n$ be some set of linear orders.
    The following are equivalent:
    \begin{enumerate}
      \item Simple majority voting over any profile of voters is always acyclic.
      \item Simple majority voting over any odd-size profile of voters
        always yields a total order.
      \item Every triple of outcomes $i,j,k\in[n]$ is \emph{acyclic}, meaning
        that $\C$ restricted to $\{i,j,k\}$ does not contain the orders
        $ijk$, $jki$, and $kij$.
      \item \label{thrmCondorcetPropsValRestr}
        Among ever triple $i,j,k\in [n]$ of distinct outcomes in $[n]$,
        one of them is either never first, never last, or never in the middle.
    \end{enumerate}

  \end{theorem}

  The above theorem is not too hard, but it's very standard so we do not prove
  it here. Some intuition/a guide to the proof is the following:
  majority voting being acyclic is equivalent to it being \emph{transitive},
  and being transitive is simply a property of triples of outcomes
  ($a>b$ and $b>c$ implies $a>c$).

  \begin{definition}
    A Condorcet domain $\C\subseteq S_n$ is any set of orders satisfying any of
    the above conditions.
  \end{definition}

\section{Preliminaries}

  If $\C$ is a Condorcet domain, theorem~\ref{thrmCondorcetProps}
  part~\ref{thrmCondorcetPropsValRestr} tells us that every triple $i,j,k$
  of outcomes falls into some ``value restriction case'':
  one of $i$, $j$, or $k$ is either
  never first, never second, or never third among $\{i,j,k\}$.
  In those cases, if $u$ is the ``restricted'' outcome among $\{i,j,k\}$,
  we write $u N_1$, $u N_2$, or $u N_3$, respectively.
  If the triple $\{i,j,k\}$ is not clear from the context, we will write
  $u N_i vw$ for $\{v,w\} = \{i,j,k\} \setminus \{u\}$.

  \begin{definition}
    A \emph{value restriction (VR) casting} on $[n]$ is a function
    $\psi$ from triples of outcomes $(i,j,k) \in {[n] \choose 3}$
    to one of the nine ``value restriction cases''
    $u N_v$ for $u\in\{i,j,k\}$ and $v\in \{1,2,3\}$.

    The \emph{corresponding domain} $\D(\psi)$ of some value restriction 
    is the collection of all orders $\sigma \in S_n$ which satisfy $\psi$ on
    every triple. That is, $\sigma\in\D(\psi)$ if for all $i,j,k \in [n]$ with
    $\psi(i,j,k) = u N_v$, $\sigma|_{i,j,k}$ does not have $u$ in position $v$.
  \end{definition}

  CDs, even closed CDs, are not in general defined
  by the VR castings they satisfy\footnote{
    For example, single crossing domains $\C$ are closed but 
    in general much smaller than $\D(\psi(\C))$.
  }. However, maximal CDs are always the corresponding domain
  of a VR casting:

  \begin{proposition}
    If $\C$ is a maximal CD, then $\C = \D(\psi)$ for some $\psi$.
  \end{proposition}
  \begin{proof}
    By theorem~\ref{thrmCondorcetProps} part~\ref{thrmCondorcetPropsValRestr},
    every CD is contained in $\D(\psi)$ for some $\psi$.
    If $\C\subsetneq \D(\psi)$, then $\C$ is not maximal.
  \end{proof}

  \begin{example}
    When $n=3$, there are three maximal CDs up to relabeling.
    They are important enough to give a name ((AND A PICTURE IN THE FINAL
    VERSION)).
  \end{example}

  \begin{proposition}
    For any $\psi$, the domain $\D(\psi)$ is median-closed.
  \end{proposition}
  \begin{proof}
    Observe that medians are preserved under taking restrictions,
    that is, for any $S\subseteq[n]$, $\med(p,q,r)|_S = 
    \med(p|_S,q|_S,r|_S)$.
    For any triple $T=\{i,j,k\}$, we have $\D|_T\subseteq\D(\psi(\{i,j,k\}))$,
    which turn out to be a \emph{representative} domain on three alternatives.
    ((THIS IS JUMPING A BIT AHEAD)).
    Thus, $\med(p|_T,q|_T,r|_T)$ will always lie in 
    $\D|_T$ when $p,q,r \in \D$, so $\med(p,q,r)$ will satisfy $\psi$ on triple
    $i,j,k$ ((A BIT INFORMAL HERE)).
  \end{proof}

  \begin{definition}
    We distinguish the two natural orders
    $\alpha = 12\ldots n$ and $\omega = n(n-1)\ldots 1$.

    The reversed order of any $\sigma\in S_n$
    is given by $\rev(\sigma) = \sigma(n)\sigma(n-1)\ldots\sigma(1)$

    A domain $\C$ is \emph{maximal width} if $\C$ contains a reversed pair.

    A domain $\C$ is \emph{normal} if $\alpha\in\C$ and $\omega\in\C$.
  \end{definition}

  We typically assume without loss of generality that any maximal width domain
  is in fact normal.

\section{Single Crossing Domains}

  \begin{theorem}
    Let $\D\subseteq S_n$ be any set of orders.
    The following are equivalent:
    \begin{enumerate}
      \item There exists an ordering $>$ on $\D$ such that,
        for all pairs of outcomes $i,j\in [n]$,
        all of the orderings $\succ$ such that $i\succ j$
        are on the same side of all of the orderings $\succ'$
        such that $j\succ' i$
    \end{enumerate}
  \end{theorem}

\section{Normal Domains and Distributive Lattices}

  Maximal width (without loss of generality, normal) CDs have a significantly
  nicer structure than general domains. For one intuitive reason why this might
  be the case, observe that if you assume $\alpha,\omega\in\C$, then
  for any triple $i<j<k$, you can only lie in one of the following four cases:
  \begin{align*}
    j N_1 ik && i N_2 jk \\
    j N_3 ik && k N_2 ij
  \end{align*}
  instead of the naive 9 possible cases.

  \begin{definition}
    The inversion set $\inv(\sigma)$ of an ordering $\succ\in S_n$ is
    the collection of (unordered) pairs $(ij)$ such that $i<j$ yet
    $j \succ i$.
  \end{definition}

  So, for example, $\inv(\alpha)=\emptyset$ and $\inv(\omega)={[n]\choose 2}$.
  In the language of SYMMETRIC CONDORCET DOMAINS BY DANILOV ET AL,
  this next proposition relates normal CDs to sublattices of the Bruhat lattice
  on $S_n$.

  \begin{proposition}
    Let $\D$ be a normal CD, closed under taking medians.
    Then $\D$ forms a distributive lattice, with join and meet operations given
    by
    \begin{align*}
      p \wedge p' := \med(\alpha,p,p') &&
      p \vee p' := \med(\omega,p,p')
    \end{align*}
    Moreover, $p \le p'$ in this lattice if and only if 
    $\inv(p)\subseteq\inv(p')$, and we have
    \begin{align*}
      \inv(p \wedge p') = \inv(p)\cap \inv(p') &&
      \inv(p \vee p') = \inv(p)\cup \inv(p')
      \tag{*}
    \end{align*}

  \end{proposition}
  \begin{proof}
    The lattice operations $\wedge$ and $\vee$ are well-defined because $\D$ is
    closed under taking medians. It's not too hard to verify that these
    operations satisfy the algebraic conditions to form a lattice.
    However, we use a less direct approach:
    essentially everything follows from the formulas (*).
    We prove the first one (the second is similar).
    If $(ij) \in\inv(p)\cap\inv(p')$, then $j$ is ranked higher than $i$ in both
    $p$ and $p'$. Thus, $j$ is again ranked higher in $p\wedge p'$, so
    $(ij)\in \inv(p\wedge p')$. On the other hand, if
    $(ij)\notin\inv(p)\cap\inv(p')$, then one of $p$ or $p'$ must rank $i$
    before $j$ (as does $\alpha$) and thus $p\wedge p'$ does as well.

    Now, the existence of $p\wedge p'$ and $p\vee p'$ satisfying (*)
    can be used to easily show that $\D$ forms a distributive lattice where
    $p\le p'$ precicely when $\inv(p)\subseteq\inv(p')$.
  \end{proof}

  Now, the first thing you should do when you find a distributive lattice is 
  look to Birkhoff's representation theorem and ask yourself:
  what are the join-irreducible elements?

  \begin{definition}
    Given a normal CD $\D$ and any pair of outcomes $i<j$, define
    \begin{align*}
      \D_{ij} := \{ \succ\in\D : j\succ i \} &&
      p_{ij} := \bigwedge \D_{ij}
    \end{align*}
  \end{definition}
  That is, $\D_{ij}$ is the set of all preferences which invert pair $(ij)$
  and $p_{ij}$ is the lowest preference which inverts $(ij)$.
  \begin{proposition}\label{thrmIrreducibles}
    % consider the map $\eta : {[n] \choose 2} \to \D$ given by
    % $\eta(ij) = p_{ij}$.
    Let $\D$ be a median-closed Condorcet domain.
    \begin{enumerate}
      \item The join irreducible elements of $\D$ are precisely the collection 
        $\{p_{ij} : (ij)\in {[n]\choose  2}\}$.
      \item \label{thrmIrreduciblesPartFormala}
        Any $p\in\D$ can be expressed as the join of elements of
        $\{p_{ij}\}$ as follows:
        \[ p = \bigvee_{(ij)\in\inv(p)} p_{ij} \]
      \item \label{thrmIrreduciblesPartInverters}
        For any $i<j$, a preference $p$ has $(ij)\in\inv(p)$ (i.e. $p$
        ranks $j$ higher than $i$) if and only if $p\ge p_{ij}$ in the lattice.
    \end{enumerate}
  \end{proposition}
  \begin{proof}
    First, we show that each $p_{ij}$ is join-irreducible. 
    Let $p_{ij} = p_1\vee p_2$, so that
    $\inv(p_{ij}) = \inv(p_1) \cup \inv(p_2)$.
    At least one of $p_1$ or $p_2$ must invert $i,j$,
    say $(ij)\in\inv(p_1)$.
    Then $p_1\in\D_{ij}$, so $p_{ij}\le p_1$.
    For any lattice, if $u = p_1\vee p_2$, then $p_1\le u$.
    Thus, we have $p_{ij}=p_1$, and $p_{ij}$ is join-irreducible.

    To show $\{p_{ij}\}_{ij}$ is actually all irreducible elements in $\D$, it
    will suffice to show \ref{thrmIrreduciblesPartFormala}.
    For, if $p\notin\{p_{ij}\}_{ij}$, then formula \ref{thrmIrreduciblesPartFormala}
    gives a way to write $p$ as the join of elements other than $p$.

    We now prove \ref{thrmIrreduciblesPartFormala}.
    Certainly $(ij)\in\inv(p_{ij})$, so
    \[ \inv(p) \subseteq \bigcup_{(ij)\in\inv(p)} \inv(p_{ij}) 
      = \inv\left(\bigvee_{(ij)\in\inv(p)} p_{ij} \right) 
      \qquad\Longrightarrow\qquad
      p \le \bigvee_{(ij)\in\inv(p)} p_{ij} \]
    On the other hand, for each $(ij)\in\inv(p)$, we have $p\in\D_{ij}$, so
    actually $p_{ij}\le p$. Thus $\bigvee_{(ij)\in\inv(p)} p_{ij}\le p$.

    To prove \ref{thrmIrreduciblesPartInverters}, notice
    that if $(ij)\in\inv(p)$, then $p\in\D_{ij}$, so $p\ge p_{ij}$.
    On the other hand, if $p\ge p_{ij}$, then $(ij)\in\inv(p)$.
  \end{proof}

  Finally, we arrive at the following:
  \begin{definition}
    For any median-closed normal domain $\D$,
    define the \emph{inversion poset} $\PP(\D)$ as follows:
    the elements of $\PP(\D)$ are the preimages in $[n]\choose 2$
    of the map $(ij)\mapsto p_{ij}$ (that is, the collection of sets of pairs
    $\{(k\ell) | p_{k\ell}=p_{ij}\}$ for some fixed $(ij)$).
    The order relations in $\PP(\D)$ are as follows:
    if $S\ni (ij)$ and $T\ni (k\ell)$, then 
    $S\le T$ if and only if $p_{ij}\le p_{k\ell}$ in $\D$.
  \end{definition}
  \begin{theorem}\label{thrmBijectDownsets}
    For any closed normal CD $\D$, the distributive lattice on $\D$ is
    isomorphic to the lattice of downward-closed sets of $\PP(\D)$.
    Moreover, the correspondance $\D\to\PowerSet{\PP(\D)}$ is
    given by $p\mapsto \{S : S\subseteq \inv(p)\}$.
  \end{theorem}
  \begin{proof}
    TODO
  \end{proof}

  \begin{example}
    \begin{enumerate}
      \item A single crossing
      \item A symmetric
      \item Fishburn's?
    \end{enumerate}
  \end{example}

\section{Normal Peak/Pit Domains}
  \begin{theorem}
    Let $\D = \D(\psi)$ be a closed normal peak/pit CD
    given by a peak/pit VR casting $\psi$.
    $\D$ is maximal if and only if the map $(ij)\mapsto p_{ij}$
    is injective (that is, every pair in $[n]\choose 2$ appears as
    its own element in $\PP(\D)$).
  \end{theorem}
  \begin{proof}
    $(\Longleftarrow)$ Consider any maximal chain from $\alpha$ to
    $\omega$ in the poset $\D$. By theorem~\ref{thrmBijectDownsets},
    this corresponds to an ordering of the set of inversions $[n]\choose 2$
    ((HOW SO??????)).
    This is actually a maximal single crossing domain.
    Observe that maximal single crossing domains have full projection onto every
    triple. Thus, I should write a ``full projection lemma'' and be done with
    this direction.

    $(\Longrightarrow)$
    Hopefully the following outline pans out.
    Suppose the map is not injective.
    Again consider a maximal chain in $\D$. This time, the chain corresponds to
    some permutation over $\PP(\D)$, i.e. an ordered list 
    $S_1,S_2,\ldots,S_\ell$ of sets of inversions.
    This corresponds to a single crossing domain.
    As we know we can show, a maximal (normal) single crossing domain actually
    inverts each pair one-by-one (i.e. it corresponds to a list of simple pairs). 
    So take a maximal single crossing domain containing the maximal chain.
    Hopefully, it will be able to show that each element of $\D = \D(\psi)$
    is contained in the ``closure'' of this maximal single crossing domain
    (using some very special properties of peak/pit domains).

  \end{proof}

  The following lemma tells us there is a canonical way to make a normal peak
  pit domain into a distributive lattice.
  \begin{proposition}
    If $\C$ is a peak/pit domain, then $\C$ can contain at most one reversed
    pair.
  \end{proposition}
  \begin{proof}
    Suppose $\C$ contains $\alpha$ and $\omega$ and a some preference 
    $p, \rev(p)\in \C$ with $p\ne \alpha, \omega$.
    There must be some triple $i<j<k$ which are neither in increasing order nor
    in reversed order in $p$.
    Up to symmetry, this leaves two cases: $p|_{\{i,j,k\}} = jik$
    and $\rev(p)|_{\{i,j,k\}} = kij$,
    or  $p|_{\{i,j,k\}} = ikj$ and $\rev(p)|_{\{i,j,k\}} = jki$.
    In either case, each of $i,j,k$ are sometimes the top and sometimes
    the last choice in $\C|_{\{i,j,k\}}$.
    Thus, this triple can satisfy no peak/pit restriction in $\C$.
  \end{proof}

  VR castings are especially useful when studying restricted classes of
  Condorcet domains. In special cases, some simple observations about which VR
  cases are possible allow us to reduce the complexity of the set of VR castings
  considerably.

  \begin{definition}
    A normal peak/pit VR casting $\psi$ is ...
    \begin{align*}
      \forall i<j<k:
      &&
      \begin{cases}
        \text{either} &  \psi(\{i,j,k\}) = j N_1 ik \\
        \text{or} &  \psi(\{i,j,k\}) = j N_3 ik
      \end{cases}
    \end{align*}
  \end{definition}
  \begin{definition}
    Given a normal peak/pit VR casting, define the inversion graph $R(\psi)$
    with vertices $[n] \choose 2$ and, for each $i<j<k$, add edges
    \begin{align*}
      \text{if $\psi(\{i,j,k\}) = j N_1 ik$ then }
      && (ij) \longrightarrow (ik) \longrightarrow (jk) \\
      \text{if $\psi(\{i,j,k\}) = j N_3 ik$ then }
      && (jk) \longrightarrow (ik) \longrightarrow (ij) \\
    \end{align*}
    % Define the inversion poset
    % (technically not a poset) $\overline{R}(\psi)$ on $[n]\choose 2$
    % as the transitive closure of the relations $p_1 > p_2$ whenever 
    % $p_1\longrightarrow p_2$ is an edge in $R(\psi)$.
  \end{definition}

  The above definition looks a bit strange, but $\overline R(\psi)$ 
  is actually defined very naturally such that the following theorem is
  essentially obvious:

  \begin{proposition}
    An order $p \in S_n$ is in $\D(\psi)$ if and only if $\inv(p)$
    is a closed set (i.e. a set closed under following directed edges)
    in the inversion graph $R(\psi)$.
  \end{proposition}
  \begin{proof}
    TODO
  \end{proof}

  We thus obtain the following correspondence with the previous section:
  \begin{proposition}
    For domain given by a normal peak/pit casting $\psi$,
    the inversion poset $\PP(\D(\psi))$ is the transitive closure of the relation
    given by $p_1 \ge p_2$ whenever $p_1\longrightarrow p_2$ is an edge in 
    $R(\psi)$.
  \end{proposition}

\section{Bad News -- Counterexamples}

  The property of maximality gives a lot of structure to Condorcet domains.
  However, this structure is complicated.

  \begin{example}
    There are maximal Condorcet domains without full projection onto some
    triple. Moreover, maximality is not preserved under restrictions.
  \end{example}

\end{document}

